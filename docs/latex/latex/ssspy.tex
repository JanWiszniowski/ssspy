%% Generated by Sphinx.
\def\sphinxdocclass{report}
\documentclass[letterpaper,10pt,english]{sphinxmanual}
\ifdefined\pdfpxdimen
   \let\sphinxpxdimen\pdfpxdimen\else\newdimen\sphinxpxdimen
\fi \sphinxpxdimen=.75bp\relax
\ifdefined\pdfimageresolution
    \pdfimageresolution= \numexpr \dimexpr1in\relax/\sphinxpxdimen\relax
\fi
%% let collapsible pdf bookmarks panel have high depth per default
\PassOptionsToPackage{bookmarksdepth=5}{hyperref}

\PassOptionsToPackage{booktabs}{sphinx}
\PassOptionsToPackage{colorrows}{sphinx}

\PassOptionsToPackage{warn}{textcomp}
\usepackage[utf8]{inputenc}
\ifdefined\DeclareUnicodeCharacter
% support both utf8 and utf8x syntaxes
  \ifdefined\DeclareUnicodeCharacterAsOptional
    \def\sphinxDUC#1{\DeclareUnicodeCharacter{"#1}}
  \else
    \let\sphinxDUC\DeclareUnicodeCharacter
  \fi
  \sphinxDUC{00A0}{\nobreakspace}
  \sphinxDUC{2500}{\sphinxunichar{2500}}
  \sphinxDUC{2502}{\sphinxunichar{2502}}
  \sphinxDUC{2514}{\sphinxunichar{2514}}
  \sphinxDUC{251C}{\sphinxunichar{251C}}
  \sphinxDUC{2572}{\textbackslash}
\fi
\usepackage{cmap}
\usepackage[T1]{fontenc}
\usepackage{amsmath,amssymb,amstext}
\usepackage{babel}



\usepackage{tgtermes}
\usepackage{tgheros}
\renewcommand{\ttdefault}{txtt}



\usepackage[Bjarne]{fncychap}
\usepackage{sphinx}

\fvset{fontsize=auto}
\usepackage{geometry}


% Include hyperref last.
\usepackage{hyperref}
% Fix anchor placement for figures with captions.
\usepackage{hypcap}% it must be loaded after hyperref.
% Set up styles of URL: it should be placed after hyperref.
\urlstyle{same}


\usepackage{sphinxmessages}
\setcounter{tocdepth}{3}
\setcounter{secnumdepth}{3}


\title{SSSPy}
\date{Feb 15, 2025}
\release{0.0.1}
\author{Jan Wiszniowski}
\newcommand{\sphinxlogo}{\vbox{}}
\renewcommand{\releasename}{Release}
\makeindex
\begin{document}

\ifdefined\shorthandoff
  \ifnum\catcode`\=\string=\active\shorthandoff{=}\fi
  \ifnum\catcode`\"=\active\shorthandoff{"}\fi
\fi

\pagestyle{empty}
\sphinxmaketitle
\pagestyle{plain}
\sphinxtableofcontents
\pagestyle{normal}
\phantomsection\label{\detokenize{index::doc}}


\sphinxAtStartPar
\sphinxstylestrong{Simple seismic signal simulation from the simple source moment time function}
\begin{quote}\begin{description}
\sphinxlineitem{Copyright}
\sphinxAtStartPar
2025 Jan Wiszniowski \sphinxhref{mailto:jwisz@igf.edu.pl}{jwisz@igf.edu.pl}

\end{description}\end{quote}

\sphinxAtStartPar
Contents:

\sphinxstepscope


\chapter{Methodology}
\label{\detokenize{description:methodology}}\label{\detokenize{description:methodology-head}}\label{\detokenize{description::doc}}
\sphinxAtStartPar
The goal of this programme was the assessment of the near and intermediate field influence on the seismic signal parameters,
like maximum displacement amplitude of specific phases.
The assessment is based on a very simple seismic signal simulation.
The assumption point source, homogeneous and isotropic medium, simple Haskell {[}\hyperlink{cite.bibliography:id13}{Haskell, 1964}{]} source time function,
and a double couple mechanism was taken.
Additionally, the Brune model {[}\hyperlink{cite.bibliography:id4}{Brune, 1970}{]}, {[}\hyperlink{cite.bibliography:id5}{Brune, 1971}{]}
and Knopoff and Gillbert {[}\hyperlink{cite.bibliography:id14}{Knopoff and Gilbert, 1959}{]}
are analysed to check the source time function impact on the estimates.
Knopoff and Gillbert source time function is tested only in the frequency domain.
The the radial component of displacement is analyzed, which partially equals the signal of the P wave
and was used for moment tensor estimation in many anthropogenic seismicity cases for the moment tensor estimation
e.g. {[}\hyperlink{cite.bibliography:id10}{Wiejacz, 1992}{]}, {[}\hyperlink{cite.bibliography:id9}{Lizurek \sphinxstyleemphasis{et al.}, 2017}{]}.


\section{Source time functions}
\label{\detokenize{description:source-time-functions}}
\sphinxAtStartPar
For simplicity the Haskell {[}\hyperlink{cite.bibliography:id13}{Haskell, 1964}{]} source time function is tested
and, additionally, for comparison, the Brune model {[}\hyperlink{cite.bibliography:id4}{Brune, 1970}{]}, {[}\hyperlink{cite.bibliography:id5}{Brune, 1971}{]}
source time function is applied and the Knopoff and Gillbert {[}\hyperlink{cite.bibliography:id14}{Knopoff and Gilbert, 1959}{]}
model only in the frequency domain.

\sphinxAtStartPar
The Haskell source time function is
\begin{equation}\label{equation:description:eq_b1}
\begin{split}M\left( t \right)= \begin{cases}
0 & \text{ for } t < 0 \\
tM_0/\tau & \text{ for } 0 \leqslant  t \leqslant \tau \\
M_0 & \text{ for } t > \tau
\end{cases},\end{split}
\end{equation}
\sphinxAtStartPar
where \(M_0\) is the moment, and \(\tau\) is the rupture time.
The alternative form of \eqref{equation:description:eq_b1} is
\begin{equation*}
\begin{split}M\left( t \right) = \frac{M_0}{\tau}\left[ tH\left( t \right)-\left( t-\tau \right)H\left( t-\tau \right) \right]\end{split}
\end{equation*}
\sphinxAtStartPar
where \(H\left( t \right)\) is Heaviside step function.

\sphinxAtStartPar
In the frequency domain, Haskell source function is
\begin{equation}\label{equation:description:eq_b2}
\begin{split}M\left( \omega \right) = \frac{M_0}{\tau\omega^2}\left[ \exp\left( -j\omega\tau \right)-1 \right],\end{split}
\end{equation}
\sphinxAtStartPar
where \(\omega=2\pi f\).

\sphinxAtStartPar
The Brune source time function is
\begin{equation}\label{equation:description:eq_b3}
\begin{split}M\left( t \right) = M_0\left[ 1 - \exp\left( -t/ \tau \right)\left( t/ \tau +1 \right) \right].\end{split}
\end{equation}
\sphinxAtStartPar
In the frequency domain, Brune source function is
\begin{equation}\label{equation:description:eq_b4}
\begin{split}M\left( \omega \right) = \frac{M_0\omega_0^2}{j\omega\left( \omega_0^2-\omega^2  \right)},\end{split}
\end{equation}
\sphinxAtStartPar
where \(\omega_0=1 / \tau\).

\sphinxAtStartPar
The simplest Knopoff and Gillbert model well displays the near and intermediate effect
in the frequency domain, because
\begin{equation}\label{equation:description:eq_b5}
\begin{split}M\left( \omega \right) = M_0.\end{split}
\end{equation}

\section{Displacement calculation}
\label{\detokenize{description:displacement-calculation}}
\sphinxAtStartPar
With assumptions that simplify the model, we use the total displacement in homogeneous and isotropic medium
caused by the point double couple formula {[}\hyperlink{cite.bibliography:id11}{Aki and Richards, 2002}{]}
\begin{equation}\label{equation:description:eq_a0}
\begin{split}\begin{matrix}
\mathbf{u}\left ( \mathbf{r},t \right ) &
=9\sin2\theta\cos\phi\mathbf{R} &
-6\left(\cos2\theta\cos\phi\mathbf{\Theta} - \cos\theta\sin\phi\mathbf{\Phi}  \right)  &
\frac{1}{4\pi\rho r^4}\int_{r/v_p}^{r/v_s}\tau M\left( t-\tau \right)d\tau  \\ &
+4 \sin2\theta\cos\phi\mathbf{R} &
-2 \left(\cos2\theta\cos\phi\mathbf{\Theta} - \cos\theta\sin\phi\mathbf{\Phi}  \right)  &
\frac{1}{4\pi\rho v_p^2 r^2}M\left( t-r/v_p \right)  \\ &
-3 \sin2\theta\cos\phi\mathbf{R} &
+3 \left(\cos2\theta\cos\phi\mathbf{\Theta} - \cos\theta\sin\phi\mathbf{\Phi}  \right)  &
\frac{1}{4\pi\rho v_s^2 r^2}M\left( t-r/v_s \right)  \\ &
+ \sin2\theta\cos\phi\mathbf{R} & &
\frac{1}{4\pi\rho v_p^3 r}\dot{M}\left( t-r/v_p \right)  \\ & &
+ \left(\cos2\theta\cos\phi\mathbf{\Theta} - \cos\theta\sin\phi\mathbf{\Phi}  \right)  &
\frac{1}{4\pi\rho v_s^3 r}\dot{M}\left( t-r/v_s \right),  \\
\end{matrix}\end{split}
\end{equation}
\sphinxAtStartPar
where \(\theta\) and \(\phi\) are ratiation angles,
\(\mathbf{R}\) is the unit vector of the source\sphinxhyphen{}receiver radial direction,
\(\mathbf{\Phi}\) is the perpendicular to the radial direction horizontal unit vector,
and \(\mathbf{\Theta}\) is the unit vector completing the coordinate system.

\sphinxAtStartPar
We will organize the formula \eqref{equation:description:eq_a0} algorithmically as follows:
\begin{equation}\label{equation:description:eq_a1}
\begin{split}\mathbf{u}\left ( \mathbf{r},t \right )=
\mathbf{u}_R\left ( \mathbf{r},t \right )+
\mathbf{u_T}\left ( \mathbf{r},t \right ),\end{split}
\end{equation}
\sphinxAtStartPar
where \(u_R\) is the radial part of the displacement, \(u_T\) is the transversal part of the displacement.
\begin{equation}\label{equation:description:eq_a2}
\begin{split}\mathbf{\mathbf{u}}_* \left(r, t \right) =
\frac{\mathbf{R}^{N*}}{4\pi\rho r^4}\int_{r/v_p}^{r/v_s}\tau M\left( t-\tau \right)d\tau
+ \left[ \frac{\mathbf{R}^{I*P}}{v_p^2} + \frac{\mathbf{R}^{I*S}}{v_s^2} \right]
\frac{1}{4\pi\rho r^2}M\left( t \right)
+\frac{\mathbf{R}^{F*}}{4\pi\rho v_*^3 r } \dot{M}\left( t \right),\end{split}
\end{equation}
\sphinxAtStartPar
where \(*\) means either a radial (\(R\)) or transversal (\(T\)) member of \eqref{equation:description:eq_a1},
\(v_*=v_p\) for radial component and \(v_*=v_s\) for transversal component.
radiation patterns of the near and intermediate fields depend on the far field radiation patterns according to
\(\mathbf{R}^{IRP}= 4\mathbf{R}^{FR}\), \(\mathbf{R}^{IRS}= -3\mathbf{R}^{FR}\),
\(\mathbf{R}^{ITP}= -2\mathbf{R}^{FT}\), \(\mathbf{R}^{ITS}= 3\mathbf{R}^{FT}\),
\(\mathbf{R}^{NR}= 9\mathbf{R}^{FT}\), \(\mathbf{R}^{NT}= -6\mathbf{R}^{FT}\).
Far field patterns exact description,
which depend on the direction angles {[}\hyperlink{cite.bibliography:id11}{Aki and Richards, 2002}{]},
has no significance for our research.

\sphinxAtStartPar
The assessment of the near field displacement required the calculation of
\(\int_{r/v_p}^{r/v_s}\tau M\left( t-\tau \right)d\tau\),
which in the time domain is the convolution of the source time function
\(M\left( t \right)\) and the function described by the formula
\begin{equation}\label{equation:description:eq_a3}
\begin{split}G\left( t \right) = t(H(t-r/v_p) - H(t-r/v_s)),\end{split}
\end{equation}
\sphinxAtStartPar
where \(H\left( t \right)\) is Heaviside step function.
In the frequency domain, the calculation of the near field displacement required the multiplication of
source complex function in the frequency domain \(M\left( \omega \right)\) and the function
\begin{equation}\label{equation:description:eq_a4}
\begin{split}G\left( \omega \right) = \frac{\left(j\omega r/v_s+1\right)exp\left(-j\omega r/v_s\right)
-\left(j\omega r/v_p+1\right)exp\left(-j\omega r/v_p\right)}{\omega ^2}\end{split}
\end{equation}
\sphinxAtStartPar
where \(\omega=2\pi f\) and \(j=\sqrt{-1}\) is.

\sphinxAtStartPar
The assessment of the far field displacement required the differentiate
of source time function.

\sphinxstepscope


\chapter{Configuration}
\label{\detokenize{configuration:configuration}}\label{\detokenize{configuration:configuration-head}}\label{\detokenize{configuration::doc}}
\sphinxAtStartPar
The configuration is kept in the Python dictionary,
where keys are case\sphinxhyphen{}sensitive strings and values depend on parameters.
They can be strings, float values, integer values, boolean values, sub\sphinxhyphen{}dictionaries, or lists.

\sphinxAtStartPar
The configuration file (example name: \sphinxcode{\sphinxupquote{config.json}}) is a file in JavaScript Object Notation (JSON)
Here is the example file:

\begin{sphinxVerbatim}[commandchars=\\\{\}]
\PYG{p}{\PYGZob{}}
  \PYG{l+s+s2}{\PYGZdq{}}\PYG{l+s+s2}{source\PYGZus{}model}\PYG{l+s+s2}{\PYGZdq{}}\PYG{p}{:} \PYG{l+s+s2}{\PYGZdq{}}\PYG{l+s+s2}{Haskell}\PYG{l+s+s2}{\PYGZdq{}}\PYG{p}{,}
  \PYG{l+s+s2}{\PYGZdq{}}\PYG{l+s+s2}{green\PYGZus{}function}\PYG{l+s+s2}{\PYGZdq{}}\PYG{p}{:} \PYG{l+s+s2}{\PYGZdq{}}\PYG{l+s+s2}{homogeneous}\PYG{l+s+s2}{\PYGZdq{}}\PYG{p}{,}
  \PYG{l+s+s2}{\PYGZdq{}}\PYG{l+s+s2}{dt}\PYG{l+s+s2}{\PYGZdq{}}\PYG{p}{:} \PYG{l+m+mf}{0.001}\PYG{p}{,}
  \PYG{l+s+s2}{\PYGZdq{}}\PYG{l+s+s2}{density}\PYG{l+s+s2}{\PYGZdq{}}\PYG{p}{:} \PYG{l+m+mi}{2700}\PYG{p}{,}
  \PYG{l+s+s2}{\PYGZdq{}}\PYG{l+s+s2}{radial\PYGZus{}radiation}\PYG{l+s+s2}{\PYGZdq{}}\PYG{p}{:} \PYG{l+m+mf}{1.0}\PYG{p}{,}
  \PYG{l+s+s2}{\PYGZdq{}}\PYG{l+s+s2}{vp}\PYG{l+s+s2}{\PYGZdq{}}\PYG{p}{:} \PYG{l+m+mf}{5000.0}\PYG{p}{,}
  \PYG{l+s+s2}{\PYGZdq{}}\PYG{l+s+s2}{vs}\PYG{l+s+s2}{\PYGZdq{}}\PYG{p}{:} \PYG{l+m+mf}{2900.0}\PYG{p}{,}
  \PYG{l+s+s2}{\PYGZdq{}}\PYG{l+s+s2}{stop\PYGZus{}simulation}\PYG{l+s+s2}{\PYGZdq{}}\PYG{p}{:} \PYG{l+s+s2}{\PYGZdq{}}\PYG{l+s+s2}{rupture\PYGZus{}time}\PYG{l+s+s2}{\PYGZdq{}}\PYG{p}{,}
  \PYG{l+s+s2}{\PYGZdq{}}\PYG{l+s+s2}{source\PYGZus{}parameters}\PYG{l+s+s2}{\PYGZdq{}}\PYG{p}{:} \PYG{p}{[}
    \PYG{p}{\PYGZob{}}\PYG{l+s+s2}{\PYGZdq{}}\PYG{l+s+s2}{moment\PYGZus{}scalar}\PYG{l+s+s2}{\PYGZdq{}}\PYG{p}{:} \PYG{l+m+mf}{1e14}\PYG{p}{,} \PYG{l+s+s2}{\PYGZdq{}}\PYG{l+s+s2}{rupture\PYGZus{}time}\PYG{l+s+s2}{\PYGZdq{}}\PYG{p}{:} \PYG{l+m+mf}{0.05}\PYG{p}{\PYGZcb{}}\PYG{p}{,}
    \PYG{p}{\PYGZob{}}\PYG{l+s+s2}{\PYGZdq{}}\PYG{l+s+s2}{moment\PYGZus{}scalar}\PYG{l+s+s2}{\PYGZdq{}}\PYG{p}{:} \PYG{l+m+mf}{1e15}\PYG{p}{,} \PYG{l+s+s2}{\PYGZdq{}}\PYG{l+s+s2}{rupture\PYGZus{}time}\PYG{l+s+s2}{\PYGZdq{}}\PYG{p}{:} \PYG{l+m+mf}{0.1}\PYG{p}{\PYGZcb{}}\PYG{p}{,}
    \PYG{p}{\PYGZob{}}\PYG{l+s+s2}{\PYGZdq{}}\PYG{l+s+s2}{moment\PYGZus{}scalar}\PYG{l+s+s2}{\PYGZdq{}}\PYG{p}{:} \PYG{l+m+mf}{1e16}\PYG{p}{,} \PYG{l+s+s2}{\PYGZdq{}}\PYG{l+s+s2}{rupture\PYGZus{}time}\PYG{l+s+s2}{\PYGZdq{}}\PYG{p}{:} \PYG{l+m+mf}{0.3}\PYG{p}{\PYGZcb{}}
  \PYG{p}{]}\PYG{p}{,}
  \PYG{l+s+s2}{\PYGZdq{}}\PYG{l+s+s2}{inversion\PYGZus{}type}\PYG{l+s+s2}{\PYGZdq{}}\PYG{p}{:} \PYG{l+s+s2}{\PYGZdq{}}\PYG{l+s+s2}{general}\PYG{l+s+s2}{\PYGZdq{}}\PYG{p}{,}
  \PYG{l+s+s2}{\PYGZdq{}}\PYG{l+s+s2}{distances}\PYG{l+s+s2}{\PYGZdq{}}\PYG{p}{:} \PYG{p}{[}\PYG{l+m+mi}{500}\PYG{p}{,} \PYG{l+m+mi}{1000}\PYG{p}{,} \PYG{l+m+mi}{2000}\PYG{p}{,} \PYG{l+m+mi}{5000}\PYG{p}{,} \PYG{l+m+mi}{10000}\PYG{p}{,} \PYG{l+m+mi}{20000}\PYG{p}{]}\PYG{p}{,}
  \PYG{l+s+s2}{\PYGZdq{}}\PYG{l+s+s2}{inventory}\PYG{l+s+s2}{\PYGZdq{}}\PYG{p}{:} \PYG{p}{\PYGZob{}}
    \PYG{l+s+s2}{\PYGZdq{}}\PYG{l+s+s2}{file\PYGZus{}name}\PYG{l+s+s2}{\PYGZdq{}}\PYG{p}{:} \PYG{l+s+s2}{\PYGZdq{}}\PYG{l+s+s2}{VN\PYGZus{}Stations.xml}\PYG{l+s+s2}{\PYGZdq{}}\PYG{p}{,}
    \PYG{l+s+s2}{\PYGZdq{}}\PYG{l+s+s2}{file\PYGZus{}format}\PYG{l+s+s2}{\PYGZdq{}}\PYG{p}{:} \PYG{l+s+s2}{\PYGZdq{}}\PYG{l+s+s2}{STATIONXML}\PYG{l+s+s2}{\PYGZdq{}}
  \PYG{p}{\PYGZcb{}}\PYG{p}{,}
  \PYG{l+s+s2}{\PYGZdq{}}\PYG{l+s+s2}{stream}\PYG{l+s+s2}{\PYGZdq{}}\PYG{p}{:} \PYG{p}{\PYGZob{}}
    \PYG{l+s+s2}{\PYGZdq{}}\PYG{l+s+s2}{source}\PYG{l+s+s2}{\PYGZdq{}}\PYG{p}{:} \PYG{l+s+s2}{\PYGZdq{}}\PYG{l+s+s2}{arclink}\PYG{l+s+s2}{\PYGZdq{}}\PYG{p}{,}
    \PYG{l+s+s2}{\PYGZdq{}}\PYG{l+s+s2}{host}\PYG{l+s+s2}{\PYGZdq{}}\PYG{p}{:} \PYG{l+s+s2}{\PYGZdq{}}\PYG{l+s+s2}{tytan.igf.edu.pl}\PYG{l+s+s2}{\PYGZdq{}}\PYG{p}{,}
    \PYG{l+s+s2}{\PYGZdq{}}\PYG{l+s+s2}{port}\PYG{l+s+s2}{\PYGZdq{}}\PYG{p}{:} \PYG{l+s+s2}{\PYGZdq{}}\PYG{l+s+s2}{18001}\PYG{l+s+s2}{\PYGZdq{}}\PYG{p}{,}
    \PYG{l+s+s2}{\PYGZdq{}}\PYG{l+s+s2}{user}\PYG{l+s+s2}{\PYGZdq{}}\PYG{p}{:} \PYG{l+s+s2}{\PYGZdq{}}\PYG{l+s+s2}{anonymous@igf.edu.pl}\PYG{l+s+s2}{\PYGZdq{}}\PYG{p}{,}
    \PYG{l+s+s2}{\PYGZdq{}}\PYG{l+s+s2}{timeout}\PYG{l+s+s2}{\PYGZdq{}}\PYG{p}{:} \PYG{l+m+mi}{300}\PYG{p}{,}
    \PYG{l+s+s2}{\PYGZdq{}}\PYG{l+s+s2}{net}\PYG{l+s+s2}{\PYGZdq{}}\PYG{p}{:} \PYG{l+s+s2}{\PYGZdq{}}\PYG{l+s+s2}{VN}\PYG{l+s+s2}{\PYGZdq{}}\PYG{p}{,}
    \PYG{l+s+s2}{\PYGZdq{}}\PYG{l+s+s2}{cache}\PYG{l+s+s2}{\PYGZdq{}} \PYG{p}{:} \PYG{l+s+s2}{\PYGZdq{}}\PYG{l+s+s2}{cache\PYGZus{}Mw}\PYG{l+s+s2}{\PYGZdq{}}
  \PYG{p}{\PYGZcb{}}
\PYG{p}{\PYGZcb{}}
\end{sphinxVerbatim}


\section{Configuration parameters}
\label{\detokenize{configuration:configuration-parameters}}
\sphinxAtStartPar
Below is the description of parameters.
\begin{quote}\begin{description}
\sphinxlineitem{source\_model}
\sphinxAtStartPar
(str) The source model name.
Two values are allowed “Haskell”, or “Brune”

\sphinxlineitem{green\_function}
\sphinxAtStartPar
(str) The Green function type  name.
So far only “homogeneous” is allowed.

\sphinxlineitem{dt}
\sphinxAtStartPar
(float) The time sampling step for calculations

\sphinxlineitem{density}
\sphinxAtStartPar
(float) The density at the source {[}kg/m\textasciicircum{}3{]}

\sphinxlineitem{radial\_radiation}
\sphinxAtStartPar
(float) The radial radiation absolute value in the far field.

\sphinxlineitem{transversal\_radiation}
\sphinxAtStartPar
(float) The transversal radiation absolute value in the far field.
This parameter is not used so far in the assessment.
Together with the radial\_radiation, using both parameters required correct calculations
according  {[}\hyperlink{cite.bibliography:id11}{Aki and Richards, 2002}{]} page 79 or  {[}\hyperlink{cite.bibliography:id12}{Gibowicz and Kijko, 1994}{]} page 192.

\sphinxlineitem{vp}
\sphinxAtStartPar
(float) The P wave velocity {[}m/s{]}.

\sphinxlineitem{vs}
\sphinxAtStartPar
(float) The S wave velocity {[}m/s{]}.

\sphinxlineitem{stop\_simulation}
\sphinxAtStartPar
(str or float) The information when stop simulation and signal visualization.
There are three possibilities: “rupture\_time” stops simulation at phase S time arrival
+ the rupture time + 0.5 s, “phase\_time” stops simulation at phase S time arrival,
where the digit value stops simulation after the defined number of seconds.

\sphinxlineitem{source\_parameters}
\sphinxAtStartPar
(list) The list of source parameters, that figures are plotted
See {\hyperref[\detokenize{configuration:source-parameters-description}]{\sphinxcrossref{\DUrole{std,std-ref}{Source parameters description}}}}.

\sphinxlineitem{inversion\_type}
\sphinxAtStartPar
The name of tensor inversion type.
It must belong to the QuakeML MTInversionType category:
\sphinxcode{\sphinxupquote{\textquotesingle{}general\textquotesingle{}}}, \sphinxcode{\sphinxupquote{\textquotesingle{}zero trace\textquotesingle{}}}, \sphinxcode{\sphinxupquote{\textquotesingle{}double couple\textquotesingle{}}}, or None.

\sphinxlineitem{inversion\_type}
\sphinxAtStartPar
The focal mechanism inversion type name for choosing the focal mechanism.

\sphinxlineitem{distances}
\sphinxAtStartPar
(list) The list of distances at which displacements plotted are plotted in each figure.

\sphinxlineitem{inventory}
\sphinxAtStartPar
(dict) The dictionary of parameters defining how to get the inventory of all stations
(see {\hyperref[\detokenize{configuration:inventory-parameters}]{\sphinxcrossref{\DUrole{std,std-ref}{Inventory parameters}}}})

\sphinxlineitem{stream}
\sphinxAtStartPar
(dict) {\hyperref[\detokenize{configuration:stream-parameters}]{\sphinxcrossref{\DUrole{std,std-ref}{Stream parameters}}}} describing how to get streams and inventories
from the seismic data server (required only if the inventory file must be created,
see. {\hyperref[\detokenize{configuration:stream-parameters}]{\sphinxcrossref{\DUrole{std,std-ref}{Stream parameters}}}})

\end{description}\end{quote}


\section{Source parameters description}
\label{\detokenize{configuration:source-parameters-description}}
\sphinxAtStartPar
The source parameters are dictionary of two items required to calculate the source.
\begin{quote}\begin{description}
\sphinxlineitem{moment\_scalar}
\sphinxAtStartPar
(float) The moment scalar of the DC seismic moment \sphinxhyphen{} \(M_0\).

\sphinxlineitem{rupture\_time}
\sphinxAtStartPar
(float) The rupture time in the case of Haskell model.
In other models it is the time parameters of the model.

\end{description}\end{quote}


\section{Inventory parameters}
\label{\detokenize{configuration:inventory-parameters}}
\sphinxAtStartPar
The \sphinxtitleref{Inventory parameters} describe how to read station inventories.
\begin{quote}\begin{description}
\sphinxlineitem{file\_name}
\sphinxAtStartPar
The file name of the inventory file (optional, default value is “inventory.xml”).
When the file doesn’t exist, the program tries to download the inventory to the file
from the server defined in {\hyperref[\detokenize{configuration:stream-parameters}]{\sphinxcrossref{\DUrole{std,std-ref}{Stream parameters}}}},

\sphinxlineitem{file\_format}
\sphinxAtStartPar
The inventory format (optional, default value is “STATIONXML”).
It is not required when the inventory file exists

\end{description}\end{quote}


\section{Stream parameters}
\label{\detokenize{configuration:stream-parameters}}\begin{quote}\begin{description}
\sphinxlineitem{source}
\sphinxAtStartPar
(str) The web server source type (required, available options “arclink”, “fdsnws”)

\sphinxlineitem{host}
\sphinxAtStartPar
(str) Host name (required)

\sphinxlineitem{port}
\sphinxAtStartPar
(int) Server port number, (optional)

\sphinxlineitem{user}
\sphinxAtStartPar
(int) User name, (required for arclink)

\sphinxlineitem{timeout}
\sphinxAtStartPar
The waiting time for the server response (optional)

\sphinxlineitem{net}
\sphinxAtStartPar
(str) The network code (required if \sphinxtitleref{stations} parameter is missing)

\sphinxlineitem{loc}
\sphinxAtStartPar
(str) The location filter (optional)

\sphinxlineitem{chan}
\sphinxAtStartPar
(str) Channels filter (optional)

\sphinxlineitem{stations}
\sphinxAtStartPar
(list(str)) list of station names. When stations names are in the form “NN.SSSS”
where “NN” is the network code and “SSSS” is the station code.
The “net” parameter can be omitted.
If stations names are in the form “SSSS”, the “net” parameter must be defined.
It is possible to define in the list individual channels in the form “NN.SSSS.LL.CCC”
where “LL” is a location code (can be empty) and “CCC” is the channel code.

\sphinxlineitem{cache}
\sphinxAtStartPar
(str) the cache directory (optional, if missing data are not cached)

\end{description}\end{quote}

\sphinxstepscope


\chapter{SSSPy API}
\label{\detokenize{api:ssspy-api}}\label{\detokenize{api::doc}}
\sphinxstepscope


\section{Executable modules}
\label{\detokenize{api_run:executable-modules}}\label{\detokenize{api_run::doc}}
\sphinxAtStartPar
Executable modules can be run. Hovewer, they can contain functions
used in other modules.
\index{module@\spxentry{module}!ssspy@\spxentry{ssspy}}\index{ssspy@\spxentry{ssspy}!module@\spxentry{module}}

\subsection{Simple seismic simulation}
\label{\detokenize{api_run:simple-seismic-simulation}}\label{\detokenize{api_run:module-ssspy}}\begin{quote}\begin{description}
\sphinxlineitem{copyright}
\sphinxAtStartPar
Jan Wiszniowski (\sphinxhref{mailto:jwisz@igf.edu.pl}{jwisz@igf.edu.pl})

\sphinxlineitem{license}
\sphinxAtStartPar
GNU Lesser General Public License, Version 3
(\sphinxurl{https://www.gnu.org/copyleft/lesser.html})

\sphinxlineitem{version 0.0.1}
\sphinxAtStartPar
2025\sphinxhyphen{}02\sphinxhyphen{}07

\end{description}\end{quote}

\sphinxAtStartPar
The program print displacements in near, far in time domain for various distances
and in separate subplots for various source parameters: scalar moment and rupture time.
\begin{quote}\begin{description}
\sphinxlineitem{call}
\sphinxAtStartPar
\sphinxstyleemphasis{python ssspy.py config.json}, where \sphinxstyleemphasis{config.json} is a configuration file
(see {\hyperref[\detokenize{configuration:configuration}]{\sphinxcrossref{\DUrole{std,std-ref}{Configuration}}}})

\end{description}\end{quote}
\index{plot\_simulations\_radial\_p() (in module ssspy)@\spxentry{plot\_simulations\_radial\_p()}\spxextra{in module ssspy}}

\begin{fulllineitems}
\phantomsection\label{\detokenize{api_run:ssspy.plot_simulations_radial_p}}
\pysigstartsignatures
\pysiglinewithargsret{\sphinxcode{\sphinxupquote{ssspy.}}\sphinxbfcode{\sphinxupquote{plot\_simulations\_radial\_p}}}{\sphinxparam{\DUrole{n}{configuration}}}{}
\pysigstopsignatures
\sphinxAtStartPar
Plots the figure containing P wave displacement in radial direction simulation
for various source parameters in separate subplots.
The configuration parameters must contain all required information for simulation and plotting.
This procedure runs, when you call ssspy program.
\begin{quote}\begin{description}
\sphinxlineitem{Parameters}
\sphinxAtStartPar
\sphinxstyleliteralstrong{\sphinxupquote{configuration}} (\sphinxstyleliteralemphasis{\sphinxupquote{dict}}) \textendash{} The configuration container of all parameters dictionary required for the program to work.

\end{description}\end{quote}

\end{fulllineitems}

\index{time\_simulate() (in module ssspy)@\spxentry{time\_simulate()}\spxextra{in module ssspy}}

\begin{fulllineitems}
\phantomsection\label{\detokenize{api_run:ssspy.time_simulate}}
\pysigstartsignatures
\pysiglinewithargsret{\sphinxcode{\sphinxupquote{ssspy.}}\sphinxbfcode{\sphinxupquote{time\_simulate}}}{\sphinxparam{\DUrole{n}{configuration}}\sphinxparamcomma \sphinxparam{\DUrole{n}{distance}}\sphinxparamcomma \sphinxparam{\DUrole{n}{source\_parameters}}\sphinxparamcomma \sphinxparam{\DUrole{n}{ax}\DUrole{o}{=}\DUrole{default_value}{None}}}{}
\pysigstopsignatures
\sphinxAtStartPar
The time\_simulate function simulate and optionally plots the displacement simulation
in near, intermediate and fal fields for given distance, and source parameters
\begin{quote}\begin{description}
\sphinxlineitem{Parameters}\begin{itemize}
\item {} 
\sphinxAtStartPar
\sphinxstyleliteralstrong{\sphinxupquote{configuration}} (\sphinxstyleliteralemphasis{\sphinxupquote{dict}}) \textendash{} The configuration container of all parameters dictionary required for the program to work.

\item {} 
\sphinxAtStartPar
\sphinxstyleliteralstrong{\sphinxupquote{distance}} (\sphinxstyleliteralemphasis{\sphinxupquote{float}}) \textendash{} The hypocentral distance in meters

\item {} 
\sphinxAtStartPar
\sphinxstyleliteralstrong{\sphinxupquote{source\_parameters}} (\sphinxstyleliteralemphasis{\sphinxupquote{dict}})

\item {} 
\sphinxAtStartPar
\sphinxstyleliteralstrong{\sphinxupquote{ax}} (\sphinxstyleliteralemphasis{\sphinxupquote{Matplotlib.Axes}}) \textendash{} An object encapsulating an individual subplot in a figure.
Missing or None parameter turn off plotting.

\end{itemize}

\sphinxlineitem{Returns}
\sphinxAtStartPar

\sphinxAtStartPar
The tuple of values:
\begin{itemize}
\item {} 
\sphinxAtStartPar
The time \(t_{max}\) from the rupture beginning, where displacement reaches the maximum value,

\item {} 
\sphinxAtStartPar
The maximum displacement,

\item {} 
\sphinxAtStartPar
The displacement in the near field at \(t_{max}\),

\item {} 
\sphinxAtStartPar
The displacement in the intermediate field at \(t_{max}\),

\item {} 
\sphinxAtStartPar
The displacement in the far field at \(t_{max}\),

\item {} 
\sphinxAtStartPar
The maximum displacement in the far field.

\end{itemize}


\sphinxlineitem{Return type}
\sphinxAtStartPar
tuple(float, float, float, float, float, float)

\end{description}\end{quote}

\end{fulllineitems}

\index{module@\spxentry{module}!ssscat@\spxentry{ssscat}}\index{ssscat@\spxentry{ssscat}!module@\spxentry{module}}

\subsection{Plot tests for the real catalog}
\label{\detokenize{api_run:plot-tests-for-the-real-catalog}}\label{\detokenize{api_run:module-ssscat}}\begin{quote}\begin{description}
\sphinxlineitem{copyright}
\sphinxAtStartPar
Anna Tymińska (\sphinxhref{mailto:atyminska@igf.edu.pl}{atyminska@igf.edu.pl})
Jan Wiszniowski (\sphinxhref{mailto:jwisz@igf.edu.pl}{jwisz@igf.edu.pl})

\sphinxlineitem{license}
\sphinxAtStartPar
GNU Lesser General Public License, Version 3
(\sphinxurl{https://www.gnu.org/copyleft/lesser.html})

\sphinxlineitem{version 0.0.1}
\sphinxAtStartPar
2025\sphinxhyphen{}02\sphinxhyphen{}07

\end{description}\end{quote}

\sphinxAtStartPar
The script facilitates a visual comparison of the behavior of far and intermediate fields for defined stations.
The script utilizes numerical methods and visualization tools provided by the NumPy and Matplotlib
libraries in Python. The script generates numerical simulations for distances (r) and time intervals
(delta\_t) defined from each station.
It computes the far field (ff) and intermediate field (fi) behaviors based on specified parameters
and radiation pattern coefficients.
The visualization is facilitated through two sets of plots: 3D visualization
and set of 2D log\sphinxhyphen{}log plots.
\begin{quote}\begin{description}
\sphinxlineitem{call}
\sphinxAtStartPar
\sphinxstyleemphasis{python ssspy.py config.json catalog.xml}, where \sphinxstyleemphasis{config.json} is a configuration file
(see {\hyperref[\detokenize{configuration:configuration}]{\sphinxcrossref{\DUrole{std,std-ref}{Configuration}}}}) and \sphinxstyleemphasis{catalog.xml} is a catalog file in QuakeML.

\end{description}\end{quote}
\index{fields\_on\_station() (in module ssscat)@\spxentry{fields\_on\_station()}\spxextra{in module ssscat}}

\begin{fulllineitems}
\phantomsection\label{\detokenize{api_run:ssscat.fields_on_station}}
\pysigstartsignatures
\pysiglinewithargsret{\sphinxcode{\sphinxupquote{ssscat.}}\sphinxbfcode{\sphinxupquote{fields\_on\_station}}}{\sphinxparam{\DUrole{n}{configuration}}\sphinxparamcomma \sphinxparam{\DUrole{n}{catalog}}}{}
\pysigstopsignatures
\sphinxAtStartPar
\sphinxstylestrong{Description required here}
\begin{quote}\begin{description}
\sphinxlineitem{Parameters}\begin{itemize}
\item {} 
\sphinxAtStartPar
\sphinxstyleliteralstrong{\sphinxupquote{configuration}} (\sphinxstyleliteralemphasis{\sphinxupquote{dict}}) \textendash{} The configuration container of all parameters dictionary required for the program to work.

\item {} 
\sphinxAtStartPar
\sphinxstyleliteralstrong{\sphinxupquote{catalog}} (\sphinxstyleliteralemphasis{\sphinxupquote{ObsPy.Catalog}}) \textendash{} The events catalog

\end{itemize}

\end{description}\end{quote}

\end{fulllineitems}


\sphinxstepscope


\section{Library modules}
\label{\detokenize{api_lib:library-modules}}\label{\detokenize{api_lib::doc}}
\sphinxAtStartPar
The library modules contains classes and function
required in ether module includin executable modules.
\index{module@\spxentry{module}!utils@\spxentry{utils}}\index{utils@\spxentry{utils}!module@\spxentry{module}}

\subsection{Utils for simple seismic simulation}
\label{\detokenize{api_lib:utils-for-simple-seismic-simulation}}\label{\detokenize{api_lib:module-utils}}\begin{quote}\begin{description}
\sphinxlineitem{copyright}
\sphinxAtStartPar
Jan Wiszniowski (\sphinxhref{mailto:jwisz@igf.edu.pl}{jwisz@igf.edu.pl})

\sphinxlineitem{license}
\sphinxAtStartPar
GNU Lesser General Public License, Version 3
(\sphinxurl{https://www.gnu.org/copyleft/lesser.html})

\sphinxlineitem{version 0.0.1}
\sphinxAtStartPar
2025\sphinxhyphen{}02\sphinxhyphen{}07

\end{description}\end{quote}
\index{SSSException@\spxentry{SSSException}}

\begin{fulllineitems}
\phantomsection\label{\detokenize{api_lib:utils.SSSException}}
\pysigstartsignatures
\pysiglinewithargsret{\sphinxbfcode{\sphinxupquote{exception\DUrole{w}{ }}}\sphinxcode{\sphinxupquote{utils.}}\sphinxbfcode{\sphinxupquote{SSSException}}}{\sphinxparam{\DUrole{n}{message}\DUrole{o}{=}\DUrole{default_value}{\textquotesingle{}other\textquotesingle{}}}}{}
\pysigstopsignatures
\sphinxAtStartPar
The simple seismic signal simulation exception class

\end{fulllineitems}

\index{extract\_amplitude\_delta() (in module utils)@\spxentry{extract\_amplitude\_delta()}\spxextra{in module utils}}

\begin{fulllineitems}
\phantomsection\label{\detokenize{api_lib:utils.extract_amplitude_delta}}
\pysigstartsignatures
\pysiglinewithargsret{\sphinxcode{\sphinxupquote{utils.}}\sphinxbfcode{\sphinxupquote{extract\_amplitude\_delta}}}{\sphinxparam{\DUrole{n}{event}}\sphinxparamcomma \sphinxparam{\DUrole{n}{station\_code}}}{}
\pysigstopsignatures
\sphinxAtStartPar
Function extracting from the event parameters
displacement amplitude and amplitude duration,
which is treated as the event rupture time,
of the station first peak amplitude with mechanism data.
\begin{quote}\begin{description}
\sphinxlineitem{Parameters}\begin{itemize}
\item {} 
\sphinxAtStartPar
\sphinxstyleliteralstrong{\sphinxupquote{event}} (\sphinxstyleliteralemphasis{\sphinxupquote{ObsPy.Event}}) \textendash{} The event object

\item {} 
\sphinxAtStartPar
\sphinxstyleliteralstrong{\sphinxupquote{station\_code}} (\sphinxstyleliteralemphasis{\sphinxupquote{(}}\sphinxstyleliteralemphasis{\sphinxupquote{str}}\sphinxstyleliteralemphasis{\sphinxupquote{)}}) \textendash{} The station code

\end{itemize}

\sphinxlineitem{Returns}
\sphinxAtStartPar
The displacement amplitude and its duration

\sphinxlineitem{Return type}
\sphinxAtStartPar
tuple(float, float)

\end{description}\end{quote}

\end{fulllineitems}

\index{module@\spxentry{module}!source\_models@\spxentry{source\_models}}\index{source\_models@\spxentry{source\_models}!module@\spxentry{module}}

\subsection{Simple seismic source models}
\label{\detokenize{api_lib:simple-seismic-source-models}}\label{\detokenize{api_lib:module-source_models}}\begin{quote}\begin{description}
\sphinxlineitem{copyright}
\sphinxAtStartPar
Jan Wiszniowski (\sphinxhref{mailto:jwisz@igf.edu.pl}{jwisz@igf.edu.pl})

\sphinxlineitem{license}
\sphinxAtStartPar
GNU Lesser General Public License, Version 3
(\sphinxurl{https://www.gnu.org/copyleft/lesser.html})

\sphinxlineitem{version 0.0.1}
\sphinxAtStartPar
2025\sphinxhyphen{}02\sphinxhyphen{}07

\end{description}\end{quote}
\index{BaseSourceModel (class in source\_models)@\spxentry{BaseSourceModel}\spxextra{class in source\_models}}

\begin{fulllineitems}
\phantomsection\label{\detokenize{api_lib:source_models.BaseSourceModel}}
\pysigstartsignatures
\pysiglinewithargsret{\sphinxbfcode{\sphinxupquote{class\DUrole{w}{ }}}\sphinxcode{\sphinxupquote{source\_models.}}\sphinxbfcode{\sphinxupquote{BaseSourceModel}}}{\sphinxparam{\DUrole{n}{source\_parameters}}}{}
\pysigstopsignatures
\end{fulllineitems}

\index{BruneSourceModel (class in source\_models)@\spxentry{BruneSourceModel}\spxextra{class in source\_models}}

\begin{fulllineitems}
\phantomsection\label{\detokenize{api_lib:source_models.BruneSourceModel}}
\pysigstartsignatures
\pysiglinewithargsret{\sphinxbfcode{\sphinxupquote{class\DUrole{w}{ }}}\sphinxcode{\sphinxupquote{source\_models.}}\sphinxbfcode{\sphinxupquote{BruneSourceModel}}}{\sphinxparam{\DUrole{n}{source\_parameters}}}{}
\pysigstopsignatures
\sphinxAtStartPar
Brune source model in the time domain is described as
\begin{equation*}
\begin{split}M\left( t \right) = M_0\left[ 1 - \exp\left( -t/ \tau \right)\left( t/ \tau +1 \right) \right],\end{split}
\end{equation*}
\sphinxAtStartPar
where \(M_0\) is the seismic moment value, \(\tau\) is the rupture time,
and \(H(t)\) is Heaviside step function.

\end{fulllineitems}

\index{HaskellSourceModel (class in source\_models)@\spxentry{HaskellSourceModel}\spxextra{class in source\_models}}

\begin{fulllineitems}
\phantomsection\label{\detokenize{api_lib:source_models.HaskellSourceModel}}
\pysigstartsignatures
\pysiglinewithargsret{\sphinxbfcode{\sphinxupquote{class\DUrole{w}{ }}}\sphinxcode{\sphinxupquote{source\_models.}}\sphinxbfcode{\sphinxupquote{HaskellSourceModel}}}{\sphinxparam{\DUrole{n}{source\_parameters}}}{}
\pysigstopsignatures
\sphinxAtStartPar
Haskell source model in the time domain is described as
\begin{equation*}
\begin{split}M\left( t \right)= \begin{cases}
0 & \text{ for } t < 0 \\
tM_0/\tau & \text{ for } 0 \leqslant  t \leqslant \tau \\
M_0 & \text{ for } t > \tau
\end{cases},\end{split}
\end{equation*}
\sphinxAtStartPar
where \(M_0\) is the seismic moment value, \(\tau\) is the rupture time,
and \(H(t)\) is Heaviside step function.

\end{fulllineitems}

\index{module@\spxentry{module}!green\_functions@\spxentry{green\_functions}}\index{green\_functions@\spxentry{green\_functions}!module@\spxentry{module}}

\subsection{Green function in the time domain}
\label{\detokenize{api_lib:green-function-in-the-time-domain}}\label{\detokenize{api_lib:module-green_functions}}\begin{quote}\begin{description}
\sphinxlineitem{copyright}
\sphinxAtStartPar
Jan Wiszniowski (\sphinxhref{mailto:jwisz@igf.edu.pl}{jwisz@igf.edu.pl})

\sphinxlineitem{license}
\sphinxAtStartPar
GNU Lesser General Public License, Version 3
(\sphinxurl{https://www.gnu.org/copyleft/lesser.html})

\sphinxlineitem{version 0.0.1}
\sphinxAtStartPar
2025\sphinxhyphen{}02\sphinxhyphen{}07

\end{description}\end{quote}
\index{BaseGreenFunction (class in green\_functions)@\spxentry{BaseGreenFunction}\spxextra{class in green\_functions}}

\begin{fulllineitems}
\phantomsection\label{\detokenize{api_lib:green_functions.BaseGreenFunction}}
\pysigstartsignatures
\pysiglinewithargsret{\sphinxbfcode{\sphinxupquote{class\DUrole{w}{ }}}\sphinxcode{\sphinxupquote{green\_functions.}}\sphinxbfcode{\sphinxupquote{BaseGreenFunction}}}{\sphinxparam{\DUrole{n}{dt}}\sphinxparamcomma \sphinxparam{\DUrole{n}{density}}\sphinxparamcomma \sphinxparam{\DUrole{n}{transversal\_radiation}\DUrole{o}{=}\DUrole{default_value}{1.0}}\sphinxparamcomma \sphinxparam{\DUrole{n}{radial\_radiation}\DUrole{o}{=}\DUrole{default_value}{1.0}}}{}
\pysigstopsignatures
\sphinxAtStartPar
The base class of Green function classes. It required in derived classes definitions of three functions:
\begin{itemize}
\item {} 
\sphinxAtStartPar
near(self, source\_model, distance, vp, vs, times)

\item {} 
\sphinxAtStartPar
intermediate(self, source\_model, distance, vp, vs, times)

\item {} 
\sphinxAtStartPar
far(self, source\_model, distance, vp, vs, times)

\end{itemize}

\sphinxAtStartPar
They return radial and transversal displacement responses.
\begin{quote}\begin{description}
\sphinxlineitem{Parameters}\begin{itemize}
\item {} 
\sphinxAtStartPar
\sphinxstyleliteralstrong{\sphinxupquote{dt}} \textendash{} The time sampling step for integration and differentiation calculations,

\item {} 
\sphinxAtStartPar
\sphinxstyleliteralstrong{\sphinxupquote{density}} \textendash{} The density at the source,

\item {} 
\sphinxAtStartPar
\sphinxstyleliteralstrong{\sphinxupquote{transversal\_radiation}} \textendash{} The transversal\_radiation pattern in the far field,

\item {} 
\sphinxAtStartPar
\sphinxstyleliteralstrong{\sphinxupquote{radial\_radiation}} \textendash{} The radial radiation pattern in the far field.

\end{itemize}

\end{description}\end{quote}
\index{far() (green\_functions.BaseGreenFunction method)@\spxentry{far()}\spxextra{green\_functions.BaseGreenFunction method}}

\begin{fulllineitems}
\phantomsection\label{\detokenize{api_lib:green_functions.BaseGreenFunction.far}}
\pysigstartsignatures
\pysiglinewithargsret{\sphinxbfcode{\sphinxupquote{abstract\DUrole{w}{ }}}\sphinxbfcode{\sphinxupquote{far}}}{\sphinxparam{\DUrole{n}{source\_model}}\sphinxparamcomma \sphinxparam{\DUrole{n}{distance}}\sphinxparamcomma \sphinxparam{\DUrole{n}{vp}}\sphinxparamcomma \sphinxparam{\DUrole{n}{vs}}\sphinxparamcomma \sphinxparam{\DUrole{n}{times}}\sphinxparamcomma \sphinxparam{\DUrole{n}{phase}\DUrole{o}{=}\DUrole{default_value}{\textquotesingle{}P\textquotesingle{}}}}{}
\pysigstopsignatures
\sphinxAtStartPar
Compute the far part of the displacement
\begin{quote}\begin{description}
\sphinxlineitem{Parameters}\begin{itemize}
\item {} 
\sphinxAtStartPar
\sphinxstyleliteralstrong{\sphinxupquote{source\_model}} \textendash{} The source model object,

\item {} 
\sphinxAtStartPar
\sphinxstyleliteralstrong{\sphinxupquote{distance}} \textendash{} The hypocentral distance,

\item {} 
\sphinxAtStartPar
\sphinxstyleliteralstrong{\sphinxupquote{vp}} \textendash{} The P wave velocity,

\item {} 
\sphinxAtStartPar
\sphinxstyleliteralstrong{\sphinxupquote{vs}} \textendash{} The S wave velocity,

\item {} 
\sphinxAtStartPar
\sphinxstyleliteralstrong{\sphinxupquote{times}} \textendash{} Time samples. The samples steps must equal dt,

\item {} 
\sphinxAtStartPar
\sphinxstyleliteralstrong{\sphinxupquote{phase}} \textendash{} The phase name: ‘P’ or ‘S’,

\end{itemize}

\sphinxlineitem{Returns}
\sphinxAtStartPar
The radial and transversal displacement in the far field,

\end{description}\end{quote}

\end{fulllineitems}

\index{intermediate() (green\_functions.BaseGreenFunction method)@\spxentry{intermediate()}\spxextra{green\_functions.BaseGreenFunction method}}

\begin{fulllineitems}
\phantomsection\label{\detokenize{api_lib:green_functions.BaseGreenFunction.intermediate}}
\pysigstartsignatures
\pysiglinewithargsret{\sphinxbfcode{\sphinxupquote{abstract\DUrole{w}{ }}}\sphinxbfcode{\sphinxupquote{intermediate}}}{\sphinxparam{\DUrole{n}{source\_model}}\sphinxparamcomma \sphinxparam{\DUrole{n}{distance}}\sphinxparamcomma \sphinxparam{\DUrole{n}{vp}}\sphinxparamcomma \sphinxparam{\DUrole{n}{vs}}\sphinxparamcomma \sphinxparam{\DUrole{n}{times}}\sphinxparamcomma \sphinxparam{\DUrole{n}{phase}\DUrole{o}{=}\DUrole{default_value}{\textquotesingle{}P\textquotesingle{}}}}{}
\pysigstopsignatures
\sphinxAtStartPar
Compute the intermediate part of the displacement
\begin{quote}\begin{description}
\sphinxlineitem{Parameters}\begin{itemize}
\item {} 
\sphinxAtStartPar
\sphinxstyleliteralstrong{\sphinxupquote{source\_model}} \textendash{} The source model object,

\item {} 
\sphinxAtStartPar
\sphinxstyleliteralstrong{\sphinxupquote{distance}} \textendash{} The hypocentral distance,

\item {} 
\sphinxAtStartPar
\sphinxstyleliteralstrong{\sphinxupquote{vp}} \textendash{} The P wave velocity,

\item {} 
\sphinxAtStartPar
\sphinxstyleliteralstrong{\sphinxupquote{vs}} \textendash{} The S wave velocity,

\item {} 
\sphinxAtStartPar
\sphinxstyleliteralstrong{\sphinxupquote{times}} \textendash{} Time samples. The samples steps must equal dt,

\item {} 
\sphinxAtStartPar
\sphinxstyleliteralstrong{\sphinxupquote{phase}} \textendash{} The phase name: ‘P’ or ‘S’,

\end{itemize}

\sphinxlineitem{Returns}
\sphinxAtStartPar
The radial and transversal displacement in the far field,

\end{description}\end{quote}

\end{fulllineitems}

\index{near() (green\_functions.BaseGreenFunction method)@\spxentry{near()}\spxextra{green\_functions.BaseGreenFunction method}}

\begin{fulllineitems}
\phantomsection\label{\detokenize{api_lib:green_functions.BaseGreenFunction.near}}
\pysigstartsignatures
\pysiglinewithargsret{\sphinxbfcode{\sphinxupquote{abstract\DUrole{w}{ }}}\sphinxbfcode{\sphinxupquote{near}}}{\sphinxparam{\DUrole{n}{source\_model}}\sphinxparamcomma \sphinxparam{\DUrole{n}{distance}}\sphinxparamcomma \sphinxparam{\DUrole{n}{vp}}\sphinxparamcomma \sphinxparam{\DUrole{n}{vs}}\sphinxparamcomma \sphinxparam{\DUrole{n}{times}}}{}
\pysigstopsignatures
\sphinxAtStartPar
Compute the near part of the displacement
\begin{quote}\begin{description}
\sphinxlineitem{Parameters}\begin{itemize}
\item {} 
\sphinxAtStartPar
\sphinxstyleliteralstrong{\sphinxupquote{source\_model}} \textendash{} The source model object,

\item {} 
\sphinxAtStartPar
\sphinxstyleliteralstrong{\sphinxupquote{distance}} \textendash{} The hypocentral distance,

\item {} 
\sphinxAtStartPar
\sphinxstyleliteralstrong{\sphinxupquote{vp}} \textendash{} The P wave velocity,

\item {} 
\sphinxAtStartPar
\sphinxstyleliteralstrong{\sphinxupquote{vs}} \textendash{} The S wave velocity,

\item {} 
\sphinxAtStartPar
\sphinxstyleliteralstrong{\sphinxupquote{times}} \textendash{} Time samples. The samples steps must equal dt,

\end{itemize}

\sphinxlineitem{Returns}
\sphinxAtStartPar
The radial and transversal displacement in the far field,

\end{description}\end{quote}

\end{fulllineitems}


\end{fulllineitems}

\index{HomogeneousGreenFunction (class in green\_functions)@\spxentry{HomogeneousGreenFunction}\spxextra{class in green\_functions}}

\begin{fulllineitems}
\phantomsection\label{\detokenize{api_lib:green_functions.HomogeneousGreenFunction}}
\pysigstartsignatures
\pysiglinewithargsret{\sphinxbfcode{\sphinxupquote{class\DUrole{w}{ }}}\sphinxcode{\sphinxupquote{green\_functions.}}\sphinxbfcode{\sphinxupquote{HomogeneousGreenFunction}}}{\sphinxparam{\DUrole{n}{dt}}\sphinxparamcomma \sphinxparam{\DUrole{n}{density}}\sphinxparamcomma \sphinxparam{\DUrole{n}{transversal\_radiation}\DUrole{o}{=}\DUrole{default_value}{1.0}}\sphinxparamcomma \sphinxparam{\DUrole{n}{radial\_radiation}\DUrole{o}{=}\DUrole{default_value}{1.0}}}{}
\pysigstopsignatures
\sphinxAtStartPar
The Green function in the homogeneous and isotropic medium
\begin{quote}\begin{description}
\sphinxlineitem{Parameters}\begin{itemize}
\item {} 
\sphinxAtStartPar
\sphinxstyleliteralstrong{\sphinxupquote{dt}} \textendash{} The time sampling step for integration and differentiation calculations

\item {} 
\sphinxAtStartPar
\sphinxstyleliteralstrong{\sphinxupquote{density}} \textendash{} The density at the source

\item {} 
\sphinxAtStartPar
\sphinxstyleliteralstrong{\sphinxupquote{transversal\_radiation}} \textendash{} The transversal\_radiation pattern in the far field

\item {} 
\sphinxAtStartPar
\sphinxstyleliteralstrong{\sphinxupquote{radial\_radiation}} \textendash{} The radial radiation pattern in the far field

\end{itemize}

\end{description}\end{quote}
\index{far() (green\_functions.HomogeneousGreenFunction method)@\spxentry{far()}\spxextra{green\_functions.HomogeneousGreenFunction method}}

\begin{fulllineitems}
\phantomsection\label{\detokenize{api_lib:green_functions.HomogeneousGreenFunction.far}}
\pysigstartsignatures
\pysiglinewithargsret{\sphinxbfcode{\sphinxupquote{far}}}{\sphinxparam{\DUrole{n}{source\_model}}\sphinxparamcomma \sphinxparam{\DUrole{n}{distance}}\sphinxparamcomma \sphinxparam{\DUrole{n}{vp}}\sphinxparamcomma \sphinxparam{\DUrole{n}{vs}}\sphinxparamcomma \sphinxparam{\DUrole{n}{times}}\sphinxparamcomma \sphinxparam{\DUrole{n}{phase}\DUrole{o}{=}\DUrole{default_value}{\textquotesingle{}P\textquotesingle{}}}}{}
\pysigstopsignatures
\sphinxAtStartPar
Compute the far part of the displacement
of the Green function in the homogeneous and isotropic medium.
\begin{equation*}
\begin{split}u_* \left(r, t \right) = \frac{R^{I*}}{4\pi\rho v^3 r } \dot{M}\left( t \right),\end{split}
\end{equation*}
\sphinxAtStartPar
where :math * means radial or transversal part, which in the case of far field is equivalent of the P or S wave,
\(u \left(r, t \right)\) is the displacement,
\(R^{F*}\) is the radiation of radial or transversal near field pattern,
\(r\) is the hypocentral distance, \(\rho\) is the density at the source,
\(v\) is the P or S velocity,
\(\dot{M}\left( t \right)\) is the time derivative of source time function.
\begin{quote}\begin{description}
\sphinxlineitem{Parameters}\begin{itemize}
\item {} 
\sphinxAtStartPar
\sphinxstyleliteralstrong{\sphinxupquote{source\_model}} \textendash{} The source model object.

\item {} 
\sphinxAtStartPar
\sphinxstyleliteralstrong{\sphinxupquote{distance}} \textendash{} The hypocentral distance

\item {} 
\sphinxAtStartPar
\sphinxstyleliteralstrong{\sphinxupquote{vp}} \textendash{} The P wave velocity.

\item {} 
\sphinxAtStartPar
\sphinxstyleliteralstrong{\sphinxupquote{vs}} \textendash{} The S wave velocity.

\item {} 
\sphinxAtStartPar
\sphinxstyleliteralstrong{\sphinxupquote{times}} \textendash{} Time samples. The samples steps must equal dt.

\item {} 
\sphinxAtStartPar
\sphinxstyleliteralstrong{\sphinxupquote{phase}} \textendash{} The phase name: ‘P’ or ‘S’

\end{itemize}

\sphinxlineitem{Returns}
\sphinxAtStartPar
The radial and transversal displacement in the near field

\end{description}\end{quote}

\end{fulllineitems}

\index{intermediate() (green\_functions.HomogeneousGreenFunction method)@\spxentry{intermediate()}\spxextra{green\_functions.HomogeneousGreenFunction method}}

\begin{fulllineitems}
\phantomsection\label{\detokenize{api_lib:green_functions.HomogeneousGreenFunction.intermediate}}
\pysigstartsignatures
\pysiglinewithargsret{\sphinxbfcode{\sphinxupquote{intermediate}}}{\sphinxparam{\DUrole{n}{source\_model}}\sphinxparamcomma \sphinxparam{\DUrole{n}{distance}}\sphinxparamcomma \sphinxparam{\DUrole{n}{vp}}\sphinxparamcomma \sphinxparam{\DUrole{n}{vs}}\sphinxparamcomma \sphinxparam{\DUrole{n}{times}}\sphinxparamcomma \sphinxparam{\DUrole{n}{phase}\DUrole{o}{=}\DUrole{default_value}{\textquotesingle{}P\textquotesingle{}}}}{}
\pysigstopsignatures
\sphinxAtStartPar
Compute the intermediate part of the displacement
of the Green function in the homogeneous and isotropic medium.
\begin{equation*}
\begin{split}u_* \left(r, t \right) = \frac{R^{I*}}{4\pi\rho v^2 r^2} M\left( t \right),\end{split}
\end{equation*}
\sphinxAtStartPar
where :math * means radial or transversal part,
\(u \left(r, t \right)\) is the displacement,
\(R^{I*}\) is the radiation of radial or transversal near field pattern,
\(r\) is the hypocentral distance, \(\rho\) is the density at the source,
\(v\) is the P or S velocity,
\(M\left( t \right)\) is the source time function.
For P wave \(R^{IR}= 4R^{FR}\) and \(R^{IT}= -2R^{FT}\) (see far field radiation),
for S wave \(R^{IR}= -3R^{FR}\) and \(R^{IT}= 3R^{FT}\) (see far field radiation).
\begin{quote}\begin{description}
\sphinxlineitem{Parameters}\begin{itemize}
\item {} 
\sphinxAtStartPar
\sphinxstyleliteralstrong{\sphinxupquote{source\_model}} \textendash{} The source model object.

\item {} 
\sphinxAtStartPar
\sphinxstyleliteralstrong{\sphinxupquote{distance}} \textendash{} The hypocentral distance

\item {} 
\sphinxAtStartPar
\sphinxstyleliteralstrong{\sphinxupquote{vp}} \textendash{} The P wave velocity.

\item {} 
\sphinxAtStartPar
\sphinxstyleliteralstrong{\sphinxupquote{vs}} \textendash{} The S wave velocity.

\item {} 
\sphinxAtStartPar
\sphinxstyleliteralstrong{\sphinxupquote{times}} \textendash{} Time samples. The samples steps must equal dt

\item {} 
\sphinxAtStartPar
\sphinxstyleliteralstrong{\sphinxupquote{phase}} \textendash{} The phase name: ‘P’ or ‘S’

\end{itemize}

\sphinxlineitem{Returns}
\sphinxAtStartPar
The radial and transversal displacement in the near field

\end{description}\end{quote}

\end{fulllineitems}

\index{near() (green\_functions.HomogeneousGreenFunction method)@\spxentry{near()}\spxextra{green\_functions.HomogeneousGreenFunction method}}

\begin{fulllineitems}
\phantomsection\label{\detokenize{api_lib:green_functions.HomogeneousGreenFunction.near}}
\pysigstartsignatures
\pysiglinewithargsret{\sphinxbfcode{\sphinxupquote{near}}}{\sphinxparam{\DUrole{n}{source\_model}}\sphinxparamcomma \sphinxparam{\DUrole{n}{distance}}\sphinxparamcomma \sphinxparam{\DUrole{n}{vp}}\sphinxparamcomma \sphinxparam{\DUrole{n}{vs}}\sphinxparamcomma \sphinxparam{\DUrole{n}{times}}}{}
\pysigstopsignatures
\sphinxAtStartPar
Compute the near part of the displacement
of the Green function in the homogeneous and isotropic medium.
\begin{equation*}
\begin{split}u_* \left(r, t \right) = \frac{R^{N*}}{4\pi\rho r^4}\int_{r/v_p}^{r/v_s}\tau M\left( t-\tau \right)d\tau,\end{split}
\end{equation*}
\sphinxAtStartPar
where * means radial or transversal part, \(u \left(r, t \right)\) is the displacement,
\(R^{N*}\) is the radiation of radial or transversal near field pattern
\(R^{NR}= 9R^{FR}\) and \(R^{NT}= -6R^{FT}\) (see far field radiation),
\(r\) is the hypocentral distance, \(\rho\) is the density at the source,
\(v_p\) and \(v_s\) are P and S velocities at the source,
\(M\left( t \right)\) is the source time function.
The integration is realised by the convolution of the source time function nad the signal
\(t(H(t-r/v_p) - H(t-r/v_s))\), where \(H(t)\) is Heaviside step function.
\begin{quote}\begin{description}
\sphinxlineitem{Parameters}\begin{itemize}
\item {} 
\sphinxAtStartPar
\sphinxstyleliteralstrong{\sphinxupquote{source\_model}} \textendash{} The source model object.

\item {} 
\sphinxAtStartPar
\sphinxstyleliteralstrong{\sphinxupquote{distance}} \textendash{} The hypocentral distance

\item {} 
\sphinxAtStartPar
\sphinxstyleliteralstrong{\sphinxupquote{vp}} \textendash{} The P wave velocity.

\item {} 
\sphinxAtStartPar
\sphinxstyleliteralstrong{\sphinxupquote{vs}} \textendash{} The S wave velocity.

\item {} 
\sphinxAtStartPar
\sphinxstyleliteralstrong{\sphinxupquote{times}} \textendash{} Time samples. The samples steps must equal dt

\end{itemize}

\sphinxlineitem{Returns}
\sphinxAtStartPar
The radial and transversal displacement in the near field

\end{description}\end{quote}

\end{fulllineitems}


\end{fulllineitems}


\sphinxstepscope


\section{Core modules}
\label{\detokenize{api_core:core-modules}}\label{\detokenize{api_core::doc}}
\sphinxAtStartPar
Core modules are used in the SSSPy package,
but are designated for more general use and use in other packages,
\index{module@\spxentry{module}!core.signal\_utils@\spxentry{core.signal\_utils}}\index{core.signal\_utils@\spxentry{core.signal\_utils}!module@\spxentry{module}}

\subsection{The waveform and inventory manipulation}
\label{\detokenize{api_core:the-waveform-and-inventory-manipulation}}\label{\detokenize{api_core:module-core.signal_utils}}\begin{quote}\begin{description}
\sphinxlineitem{copyright}
\sphinxAtStartPar
Jan Wiszniowski (\sphinxhref{mailto:jwisz@igf.edu.pl}{jwisz@igf.edu.pl})

\sphinxlineitem{license}
\sphinxAtStartPar
GNU Lesser General Public License, Version 3
(\sphinxurl{https://www.gnu.org/copyleft/lesser.html})

\sphinxlineitem{version 0.0.1}
\sphinxAtStartPar
2025\sphinxhyphen{}01\sphinxhyphen{}15

\end{description}\end{quote}
\index{Cache (class in core.signal\_utils)@\spxentry{Cache}\spxextra{class in core.signal\_utils}}

\begin{fulllineitems}
\phantomsection\label{\detokenize{api_core:core.signal_utils.Cache}}
\pysigstartsignatures
\pysiglinewithargsret{\sphinxbfcode{\sphinxupquote{class\DUrole{w}{ }}}\sphinxcode{\sphinxupquote{core.signal\_utils.}}\sphinxbfcode{\sphinxupquote{Cache}}}{\sphinxparam{\DUrole{n}{configuration}}\sphinxparamcomma \sphinxparam{\DUrole{n}{file\_name}}}{}
\pysigstopsignatures
\sphinxAtStartPar
The cache class for manipulating the cache metadata
\begin{quote}\begin{description}
\sphinxlineitem{Parameters}\begin{itemize}
\item {} 
\sphinxAtStartPar
\sphinxstyleliteralstrong{\sphinxupquote{configuration}} (\sphinxstyleliteralemphasis{\sphinxupquote{dict}}) \textendash{} The container of general seismic processing configuration.
The required parameter is a cache path kept in the ‘cache’.

\item {} 
\sphinxAtStartPar
\sphinxstyleliteralstrong{\sphinxupquote{file\_name}} (\sphinxstyleliteralemphasis{\sphinxupquote{str}}) \textendash{} The cache metadata file name

\end{itemize}

\end{description}\end{quote}
\index{backup() (core.signal\_utils.Cache method)@\spxentry{backup()}\spxextra{core.signal\_utils.Cache method}}

\begin{fulllineitems}
\phantomsection\label{\detokenize{api_core:core.signal_utils.Cache.backup}}
\pysigstartsignatures
\pysiglinewithargsret{\sphinxbfcode{\sphinxupquote{backup}}}{}{}
\pysigstopsignatures
\sphinxAtStartPar
Backs up the cache metadata. Saves to the JSON file.
\begin{quote}\begin{description}
\sphinxlineitem{Returns}
\sphinxAtStartPar
None

\end{description}\end{quote}

\end{fulllineitems}


\end{fulllineitems}

\index{SignalException@\spxentry{SignalException}}

\begin{fulllineitems}
\phantomsection\label{\detokenize{api_core:core.signal_utils.SignalException}}
\pysigstartsignatures
\pysiglinewithargsret{\sphinxbfcode{\sphinxupquote{exception\DUrole{w}{ }}}\sphinxcode{\sphinxupquote{core.signal\_utils.}}\sphinxbfcode{\sphinxupquote{SignalException}}}{\sphinxparam{\DUrole{n}{message}\DUrole{o}{=}\DUrole{default_value}{\textquotesingle{}other\textquotesingle{}}}}{}
\pysigstopsignatures
\end{fulllineitems}

\index{StreamLoader (class in core.signal\_utils)@\spxentry{StreamLoader}\spxextra{class in core.signal\_utils}}

\begin{fulllineitems}
\phantomsection\label{\detokenize{api_core:core.signal_utils.StreamLoader}}
\pysigstartsignatures
\pysiglinewithargsret{\sphinxbfcode{\sphinxupquote{class\DUrole{w}{ }}}\sphinxcode{\sphinxupquote{core.signal\_utils.}}\sphinxbfcode{\sphinxupquote{StreamLoader}}}{\sphinxparam{\DUrole{n}{configuration}}\sphinxparamcomma \sphinxparam{\DUrole{n}{preprocess}\DUrole{o}{=}\DUrole{default_value}{None}}}{}
\pysigstopsignatures
\sphinxAtStartPar
The stream loader loads seismic waveforms from servers ArcLink or FDSNWS
and process data initially. The loaded and processed data can be kept on local disc
in the cache directory for increase the reloading speed.
\begin{quote}\begin{description}
\sphinxlineitem{Parameters}\begin{itemize}
\item {} 
\sphinxAtStartPar
\sphinxstyleliteralstrong{\sphinxupquote{configuration}} (\sphinxstyleliteralemphasis{\sphinxupquote{dict}}) \textendash{} The container of general seismic processing configuration.
The required parameters are kept in the ‘stream’ sub\sphinxhyphen{}dictionary:

\item {} 
\sphinxAtStartPar
\sphinxstyleliteralstrong{\sphinxupquote{preprocess}} ({\hyperref[\detokenize{api_core:core.signal_utils.StreamPreprocessing}]{\sphinxcrossref{\sphinxstyleliteralemphasis{\sphinxupquote{StreamPreprocessing}}}}})

\end{itemize}

\end{description}\end{quote}

\sphinxAtStartPar
\sphinxstylestrong{The parameters present in the ‘stream’ sub\sphinxhyphen{}dictionary:}
\begin{quote}\begin{description}
\sphinxlineitem{Source}
\sphinxAtStartPar
The waveforms source. Available options are ‘arclink’ or ‘fdsnws’ (required)

\sphinxlineitem{Host}
\sphinxAtStartPar
The server host

\sphinxlineitem{Port}
\sphinxAtStartPar
The server port

\sphinxlineitem{User}
\sphinxAtStartPar
The request user id (if required)

\sphinxlineitem{Password}
\sphinxAtStartPar
The request password (if required)

\sphinxlineitem{Timeout}
\sphinxAtStartPar
The downloading timeout limit

\sphinxlineitem{Net}
\sphinxAtStartPar
The default network name.

\sphinxlineitem{Sta}
\sphinxAtStartPar
The default station name.

\sphinxlineitem{Loc}
\sphinxAtStartPar
The default location name.

\sphinxlineitem{Chan}
\sphinxAtStartPar
The default channel name.

\sphinxlineitem{Cache}
\sphinxAtStartPar
The cache directory. In the cache directory are kept all downloaded and preprocessed waveform files
and the file ‘loaded\_signals.json’ containing info

\sphinxlineitem{Stations}
\sphinxAtStartPar
The default request station list

\end{description}\end{quote}
\index{download() (core.signal\_utils.StreamLoader method)@\spxentry{download()}\spxextra{core.signal\_utils.StreamLoader method}}

\begin{fulllineitems}
\phantomsection\label{\detokenize{api_core:core.signal_utils.StreamLoader.download}}
\pysigstartsignatures
\pysiglinewithargsret{\sphinxbfcode{\sphinxupquote{download}}}{\sphinxparam{\DUrole{n}{begin\_time}}\sphinxparamcomma \sphinxparam{\DUrole{n}{end\_time}}\sphinxparamcomma \sphinxparam{\DUrole{n}{event\_id}}\sphinxparamcomma \sphinxparam{\DUrole{n}{new\_file\_name}\DUrole{o}{=}\DUrole{default_value}{None}}}{}
\pysigstopsignatures
\sphinxAtStartPar
Downloads the stream from the seismic data sever with optional caching.
\begin{quote}\begin{description}
\sphinxlineitem{Parameters}\begin{itemize}
\item {} 
\sphinxAtStartPar
\sphinxstyleliteralstrong{\sphinxupquote{new\_file\_name}} (\sphinxstyleliteralemphasis{\sphinxupquote{str}}) \textendash{} The proposed name of the file tobe stored in the cache.
If it is missing the unique random name is generated.

\item {} 
\sphinxAtStartPar
\sphinxstyleliteralstrong{\sphinxupquote{begin\_time}} (\sphinxstyleliteralemphasis{\sphinxupquote{ObsPy.UTCDateTime}}) \textendash{} The begin time of waveforms

\item {} 
\sphinxAtStartPar
\sphinxstyleliteralstrong{\sphinxupquote{end\_time}} (\sphinxstyleliteralemphasis{\sphinxupquote{ObsPy.UTCDateTime}}) \textendash{} The end time of waveforms

\item {} 
\sphinxAtStartPar
\sphinxstyleliteralstrong{\sphinxupquote{event\_id}} (\sphinxstyleliteralemphasis{\sphinxupquote{str}}) \textendash{} The event id, but it can be any string defining the stream request,
which can identify the data in case of repeated inquiry.

\end{itemize}

\sphinxlineitem{Returns}
\sphinxAtStartPar
The requested stream or None if it can not be downloaded

\sphinxlineitem{Return type}
\sphinxAtStartPar
ObsPy.Stream

\end{description}\end{quote}

\end{fulllineitems}

\index{exist\_file() (core.signal\_utils.StreamLoader method)@\spxentry{exist\_file()}\spxextra{core.signal\_utils.StreamLoader method}}

\begin{fulllineitems}
\phantomsection\label{\detokenize{api_core:core.signal_utils.StreamLoader.exist_file}}
\pysigstartsignatures
\pysiglinewithargsret{\sphinxbfcode{\sphinxupquote{exist\_file}}}{\sphinxparam{\DUrole{n}{begin\_time}}\sphinxparamcomma \sphinxparam{\DUrole{n}{end\_time}}\sphinxparamcomma \sphinxparam{\DUrole{n}{event\_id}}}{}
\pysigstopsignatures
\sphinxAtStartPar
The method checks if the requested waveform exists. A few conditions are checked.
First it checks if the cache exists. Then checks if event id exists.
The requested period must include in the existing file period.
The requested station list must include in the existing file station list.
The preprocessor name must be the same.
\begin{quote}\begin{description}
\sphinxlineitem{Parameters}\begin{itemize}
\item {} 
\sphinxAtStartPar
\sphinxstyleliteralstrong{\sphinxupquote{begin\_time}} (\sphinxstyleliteralemphasis{\sphinxupquote{ObsPy.UTCDateTime}}) \textendash{} The requested waveforms begin time

\item {} 
\sphinxAtStartPar
\sphinxstyleliteralstrong{\sphinxupquote{end\_time}} (\sphinxstyleliteralemphasis{\sphinxupquote{ObsPy.UTCDateTime}}) \textendash{} The requested waveforms begin time

\item {} 
\sphinxAtStartPar
\sphinxstyleliteralstrong{\sphinxupquote{event\_id}} (\sphinxstyleliteralemphasis{\sphinxupquote{str}}) \textendash{} The request event id. It can be the event id that the waveforms are associated
or any string that identify the request.

\end{itemize}

\sphinxlineitem{Returns}
\sphinxAtStartPar
The parameters of existed file or None,
if the function can not fit request to existing files list

\sphinxlineitem{Return type}
\sphinxAtStartPar
dict

\end{description}\end{quote}

\end{fulllineitems}

\index{get\_signal() (core.signal\_utils.StreamLoader method)@\spxentry{get\_signal()}\spxextra{core.signal\_utils.StreamLoader method}}

\begin{fulllineitems}
\phantomsection\label{\detokenize{api_core:core.signal_utils.StreamLoader.get_signal}}
\pysigstartsignatures
\pysiglinewithargsret{\sphinxbfcode{\sphinxupquote{get\_signal}}}{\sphinxparam{\DUrole{n}{begin\_time}}\sphinxparamcomma \sphinxparam{\DUrole{n}{end\_time}}\sphinxparamcomma \sphinxparam{\DUrole{n}{event\_id}\DUrole{o}{=}\DUrole{default_value}{None}}\sphinxparamcomma \sphinxparam{\DUrole{n}{stations}\DUrole{o}{=}\DUrole{default_value}{None}}\sphinxparamcomma \sphinxparam{\DUrole{n}{new\_file\_name}\DUrole{o}{=}\DUrole{default_value}{None}}}{}
\pysigstopsignatures
\sphinxAtStartPar
Provides seismic signal waveform based on request.
If matching the request file exist in the cache it reads signal from the file,
otherwise download from the seismic waveforms’ server.
\begin{quote}\begin{description}
\sphinxlineitem{Parameters}\begin{itemize}
\item {} 
\sphinxAtStartPar
\sphinxstyleliteralstrong{\sphinxupquote{begin\_time}} (\sphinxstyleliteralemphasis{\sphinxupquote{ObsPy.UTCDateTime}}) \textendash{} The requested waveforms begin time

\item {} 
\sphinxAtStartPar
\sphinxstyleliteralstrong{\sphinxupquote{end\_time}} (\sphinxstyleliteralemphasis{\sphinxupquote{ObsPy.UTCDateTime}}) \textendash{} The requested waveforms begin time

\item {} 
\sphinxAtStartPar
\sphinxstyleliteralstrong{\sphinxupquote{event\_id}} (\sphinxstyleliteralemphasis{\sphinxupquote{str}}) \textendash{} The request event id. It can be the event id that the waveforms are associated
or any string that identify the request. (optional If missing waveform is only downloaded from the server)

\item {} 
\sphinxAtStartPar
\sphinxstyleliteralstrong{\sphinxupquote{stations}} (\sphinxstyleliteralemphasis{\sphinxupquote{list}}\sphinxstyleliteralemphasis{\sphinxupquote{(}}\sphinxstyleliteralemphasis{\sphinxupquote{str}}\sphinxstyleliteralemphasis{\sphinxupquote{)}}) \textendash{} The request stations list.
(optional) If it is missing the station list from the configuration is checked.

\item {} 
\sphinxAtStartPar
\sphinxstyleliteralstrong{\sphinxupquote{new\_file\_name}} (\sphinxstyleliteralemphasis{\sphinxupquote{str}}) \textendash{} The name of a file in the cache.
(optional) If missing the unique file name is generated.

\end{itemize}

\sphinxlineitem{Returns}
\sphinxAtStartPar
The waveform stream. Return None if it can not (or could not) download waveforms.

\sphinxlineitem{Return type}
\sphinxAtStartPar
ObsPy.Stream

\end{description}\end{quote}

\end{fulllineitems}


\end{fulllineitems}

\index{StreamPreprocessing (class in core.signal\_utils)@\spxentry{StreamPreprocessing}\spxextra{class in core.signal\_utils}}

\begin{fulllineitems}
\phantomsection\label{\detokenize{api_core:core.signal_utils.StreamPreprocessing}}
\pysigstartsignatures
\pysiglinewithargsret{\sphinxbfcode{\sphinxupquote{class\DUrole{w}{ }}}\sphinxcode{\sphinxupquote{core.signal\_utils.}}\sphinxbfcode{\sphinxupquote{StreamPreprocessing}}}{\sphinxparam{\DUrole{n}{name}}}{}
\pysigstopsignatures
\sphinxAtStartPar
The base class of streams preprocessing
\begin{quote}\begin{description}
\sphinxlineitem{Parameters}
\sphinxAtStartPar
\sphinxstyleliteralstrong{\sphinxupquote{name}} (\sphinxstyleliteralemphasis{\sphinxupquote{str}}) \textendash{} The name of the preprocessing

\end{description}\end{quote}

\end{fulllineitems}

\index{get\_inventory() (in module core.signal\_utils)@\spxentry{get\_inventory()}\spxextra{in module core.signal\_utils}}

\begin{fulllineitems}
\phantomsection\label{\detokenize{api_core:core.signal_utils.get_inventory}}
\pysigstartsignatures
\pysiglinewithargsret{\sphinxcode{\sphinxupquote{core.signal\_utils.}}\sphinxbfcode{\sphinxupquote{get\_inventory}}}{\sphinxparam{\DUrole{n}{sta\_name}}\sphinxparamcomma \sphinxparam{\DUrole{n}{date}}\sphinxparamcomma \sphinxparam{\DUrole{n}{inventory}}}{}
\pysigstopsignatures
\sphinxAtStartPar
Extracts inventory for the station.
\begin{quote}\begin{description}
\sphinxlineitem{Parameters}\begin{itemize}
\item {} 
\sphinxAtStartPar
\sphinxstyleliteralstrong{\sphinxupquote{sta\_name}} (\sphinxstyleliteralemphasis{\sphinxupquote{str}}) \textendash{} The station name as the string in the form ‘NN.SSS’,
where ‘NN’ is the network code and ‘SSS’ is the station code.

\item {} 
\sphinxAtStartPar
\sphinxstyleliteralstrong{\sphinxupquote{date}} (\sphinxstyleliteralemphasis{\sphinxupquote{ObsPy.UTCDateTime}}) \textendash{} The date of the inventory

\item {} 
\sphinxAtStartPar
\sphinxstyleliteralstrong{\sphinxupquote{inventory}} (\sphinxstyleliteralemphasis{\sphinxupquote{ObsPy.Inventory}}) \textendash{} The inventory of all stations

\end{itemize}

\sphinxlineitem{Returns}
\sphinxAtStartPar
The inventory of the station

\sphinxlineitem{Return type}
\sphinxAtStartPar
ObsPy.Inventory

\end{description}\end{quote}

\end{fulllineitems}

\index{load\_inventory() (in module core.signal\_utils)@\spxentry{load\_inventory()}\spxextra{in module core.signal\_utils}}

\begin{fulllineitems}
\phantomsection\label{\detokenize{api_core:core.signal_utils.load_inventory}}
\pysigstartsignatures
\pysiglinewithargsret{\sphinxcode{\sphinxupquote{core.signal\_utils.}}\sphinxbfcode{\sphinxupquote{load\_inventory}}}{\sphinxparam{\DUrole{n}{configuration}}}{}
\pysigstopsignatures
\sphinxAtStartPar
Loads inventory from the file. The file name and format are in ‘inventory’ configuration.
If inventory file is missing the inventory is downloaded from the waveform server,
which configuration is in the ‘stream’ sub\sphinxhyphen{}dictionary.
\begin{quote}\begin{description}
\sphinxlineitem{Parameters}
\sphinxAtStartPar
\sphinxstyleliteralstrong{\sphinxupquote{configuration}} (\sphinxstyleliteralemphasis{\sphinxupquote{dict}}) \textendash{} The container of general seismic processing configuration.
The required parameters are kept in the ‘inventory’ sub\sphinxhyphen{}dictionary.

\sphinxlineitem{Returns}
\sphinxAtStartPar
The inventory

\sphinxlineitem{Return type}
\sphinxAtStartPar
ObsPy.Inventory

\end{description}\end{quote}

\sphinxAtStartPar
\sphinxstylestrong{The parameters present in the ‘inventory’ sub\sphinxhyphen{}dictionary:}
\begin{quote}\begin{description}
\sphinxlineitem{File\_name}
\sphinxAtStartPar
The inventory file name. (optional, default is ‘inventory.xml’)

\sphinxlineitem{File\_format}
\sphinxAtStartPar
The format of the inventory file name. (optional, default is ‘STATIONXML’)

\end{description}\end{quote}

\end{fulllineitems}

\index{module@\spxentry{module}!core.utils@\spxentry{core.utils}}\index{core.utils@\spxentry{core.utils}!module@\spxentry{module}}

\subsection{Commonly used utils for seismic data processing be the seismic processing in Python packages}
\label{\detokenize{api_core:commonly-used-utils-for-seismic-data-processing-be-the-seismic-processing-in-python-packages}}\label{\detokenize{api_core:module-core.utils}}\begin{quote}\begin{description}
\sphinxlineitem{copyright}
\sphinxAtStartPar
Jan Wiszniowski (\sphinxhref{mailto:jwisz@igf.edu.pl}{jwisz@igf.edu.pl})

\sphinxlineitem{license}
\sphinxAtStartPar
GNU Lesser General Public License, Version 3
(\sphinxurl{https://www.gnu.org/copyleft/lesser.html})

\sphinxlineitem{version 0.0.1}
\sphinxAtStartPar
2024\sphinxhyphen{}11\sphinxhyphen{}07

\end{description}\end{quote}
\index{ExtremeTraceValues (class in core.utils)@\spxentry{ExtremeTraceValues}\spxextra{class in core.utils}}

\begin{fulllineitems}
\phantomsection\label{\detokenize{api_core:core.utils.ExtremeTraceValues}}
\pysigstartsignatures
\pysiglinewithargsret{\sphinxbfcode{\sphinxupquote{class\DUrole{w}{ }}}\sphinxcode{\sphinxupquote{core.utils.}}\sphinxbfcode{\sphinxupquote{ExtremeTraceValues}}}{\sphinxparam{\DUrole{n}{trace}}\sphinxparamcomma \sphinxparam{\DUrole{n}{begin\_time}\DUrole{o}{=}\DUrole{default_value}{None}}\sphinxparamcomma \sphinxparam{\DUrole{n}{end\_time}\DUrole{o}{=}\DUrole{default_value}{None}}}{}
\pysigstopsignatures
\sphinxAtStartPar
Class that assess the extreme trace values: maximum, minimum, and absolute maximum value
\begin{quote}\begin{description}
\sphinxlineitem{Parameters}\begin{itemize}
\item {} 
\sphinxAtStartPar
\sphinxstyleliteralstrong{\sphinxupquote{trace}} (\sphinxstyleliteralemphasis{\sphinxupquote{ObsPy.Trace}}) \textendash{} The processed trace

\item {} 
\sphinxAtStartPar
\sphinxstyleliteralstrong{\sphinxupquote{begin\_time}} (\sphinxstyleliteralemphasis{\sphinxupquote{ObsPy.UTCDateTime}}) \textendash{} It limits the period, where a process is performed.
If begin\_time is not defined or it is earlier than the beginning of the trace
the process is performed from the beginning of the trace

\item {} 
\sphinxAtStartPar
\sphinxstyleliteralstrong{\sphinxupquote{end\_time}} (\sphinxstyleliteralemphasis{\sphinxupquote{ObsPy.UTCDateTime}}) \textendash{} It limits the period, where a process is performed.
If end\_time is not defined or it is later than the end of the trace,
the process is performed to the end of the trace

\end{itemize}

\end{description}\end{quote}

\sphinxAtStartPar
\sphinxstylestrong{Class variables}
\begin{quote}\begin{description}
\sphinxlineitem{Data}
\sphinxAtStartPar
The optionally cut to time limits data. The data are not a new array but subarray of the Trace data

\sphinxlineitem{Start\_time}
\sphinxAtStartPar
The time of the first data sample.

\sphinxlineitem{End\_time}
\sphinxAtStartPar
The time of the next sample after the last data sample.
It differs from the ObsPy trace end\_time, which points to the last sample of the trace

\sphinxlineitem{Max\_value}
\sphinxAtStartPar
Maximum data value

\sphinxlineitem{Max\_value}
\sphinxAtStartPar
Minimum data value

\sphinxlineitem{Abs\_max}
\sphinxAtStartPar
Absolute maximum value.
abs\_max = max(abs(min\_value), abs(max\_value))

\end{description}\end{quote}

\end{fulllineitems}

\index{IndexTrace (class in core.utils)@\spxentry{IndexTrace}\spxextra{class in core.utils}}

\begin{fulllineitems}
\phantomsection\label{\detokenize{api_core:core.utils.IndexTrace}}
\pysigstartsignatures
\pysiglinewithargsret{\sphinxbfcode{\sphinxupquote{class\DUrole{w}{ }}}\sphinxcode{\sphinxupquote{core.utils.}}\sphinxbfcode{\sphinxupquote{IndexTrace}}}{\sphinxparam{\DUrole{n}{trace}}\sphinxparamcomma \sphinxparam{\DUrole{n}{begin\_time}\DUrole{o}{=}\DUrole{default_value}{None}}\sphinxparamcomma \sphinxparam{\DUrole{n}{end\_time}\DUrole{o}{=}\DUrole{default_value}{None}}}{}
\pysigstopsignatures
\sphinxAtStartPar
Class for operating directly on time limited part of trace data
\begin{quote}\begin{description}
\sphinxlineitem{Parameters}\begin{itemize}
\item {} 
\sphinxAtStartPar
\sphinxstyleliteralstrong{\sphinxupquote{trace}} (\sphinxstyleliteralemphasis{\sphinxupquote{ObsPy.Trace}}) \textendash{} The processed trace

\item {} 
\sphinxAtStartPar
\sphinxstyleliteralstrong{\sphinxupquote{begin\_time}} (\sphinxstyleliteralemphasis{\sphinxupquote{ObsPy.UTCDateTime}}) \textendash{} It limits the period, where a process is performed.
If begin\_time is not defined or it is earlier than the beginning of the trace
the process is performed from the beginning of the trace

\item {} 
\sphinxAtStartPar
\sphinxstyleliteralstrong{\sphinxupquote{end\_time}} (\sphinxstyleliteralemphasis{\sphinxupquote{ObsPy.UTCDateTime}}) \textendash{} It limits the period, where a process is performed.
If end\_time is not defined or it is later than the end of the trace,
the process is performed to the end of the trace

\end{itemize}

\end{description}\end{quote}

\sphinxAtStartPar
\sphinxstylestrong{Class variables}
\begin{quote}\begin{description}
\sphinxlineitem{Start\_time}
\sphinxAtStartPar
The time of the first data sample index

\sphinxlineitem{End\_time}
\sphinxAtStartPar
The time of the next sample after the last data sample.
It differs from the ObsPy trace end\_time, which points to the last sample of the trace

\sphinxlineitem{Begin\_idx}
\sphinxAtStartPar
The first data sample index

\sphinxlineitem{End\_idx}
\sphinxAtStartPar
The last data sample index + 1

\end{description}\end{quote}

\sphinxAtStartPar
Example:

\begin{sphinxVerbatim}[commandchars=\\\{\}]
\PYG{o}{\PYGZgt{}\PYGZgt{}} \PYG{k+kn}{from} \PYG{n+nn}{utils} \PYG{k+kn}{import} \PYG{n}{IndexTrace}
\PYG{o}{\PYGZgt{}\PYGZgt{}} \PYG{k+kn}{from} \PYG{n+nn}{obspy}\PYG{n+nn}{.}\PYG{n+nn}{core}\PYG{n+nn}{.}\PYG{n+nn}{utcdatetime} \PYG{k+kn}{import} \PYG{n}{UTCDateTime}
\PYG{o}{\PYGZgt{}\PYGZgt{}} \PYG{n}{t1} \PYG{o}{=} \PYG{n}{UTCDateTime}\PYG{p}{(}\PYG{l+m+mi}{2024}\PYG{p}{,} \PYG{l+m+mi}{1}\PYG{p}{,} \PYG{l+m+mi}{3}\PYG{p}{,} \PYG{l+m+mi}{8}\PYG{p}{,} \PYG{l+m+mi}{28}\PYG{p}{,} \PYG{l+m+mi}{00}\PYG{p}{)}
\PYG{o}{\PYGZgt{}\PYGZgt{}} \PYG{n}{t2} \PYG{o}{=} \PYG{n}{UTCDateTime}\PYG{p}{(}\PYG{l+m+mi}{2024}\PYG{p}{,} \PYG{l+m+mi}{1}\PYG{p}{,} \PYG{l+m+mi}{3}\PYG{p}{,} \PYG{l+m+mi}{8}\PYG{p}{,} \PYG{l+m+mi}{29}\PYG{p}{,} \PYG{l+m+mi}{00}\PYG{p}{)}
\PYG{o}{\PYGZgt{}\PYGZgt{}} \PYG{n}{st} \PYG{o}{=} \PYG{n}{read}\PYG{p}{(}\PYG{l+s+s1}{\PYGZsq{}}\PYG{l+s+s1}{test.msd}\PYG{l+s+s1}{\PYGZsq{}}\PYG{p}{)}
\PYG{o}{\PYGZgt{}\PYGZgt{}} \PYG{n}{indexes} \PYG{o}{=} \PYG{n}{IndexTrace}\PYG{p}{(}\PYG{n}{st}\PYG{p}{[}\PYG{l+m+mi}{1}\PYG{p}{]}\PYG{p}{,} \PYG{n}{begin\PYGZus{}time}\PYG{o}{=}\PYG{n}{t1}\PYG{p}{,} \PYG{n}{end\PYGZus{}time}\PYG{o}{=}\PYG{n}{t2}
\PYG{o}{\PYGZgt{}\PYGZgt{}} \PYG{k}{for} \PYG{n}{idx} \PYG{o+ow}{in} \PYG{n+nb}{range}\PYG{p}{(}\PYG{n}{indexes}\PYG{o}{.}\PYG{n}{begin\PYGZus{}idx}\PYG{p}{,} \PYG{n}{indexes}\PYG{o}{.}\PYG{n}{end\PYGZus{}idx}\PYG{p}{)}\PYG{p}{:}
\PYG{o}{.}\PYG{o}{.}\PYG{o}{.} \PYG{k}{pass}
\end{sphinxVerbatim}

\end{fulllineitems}

\index{ProcessTrace (class in core.utils)@\spxentry{ProcessTrace}\spxextra{class in core.utils}}

\begin{fulllineitems}
\phantomsection\label{\detokenize{api_core:core.utils.ProcessTrace}}
\pysigstartsignatures
\pysiglinewithargsret{\sphinxbfcode{\sphinxupquote{class\DUrole{w}{ }}}\sphinxcode{\sphinxupquote{core.utils.}}\sphinxbfcode{\sphinxupquote{ProcessTrace}}}{\sphinxparam{\DUrole{n}{trace}}\sphinxparamcomma \sphinxparam{\DUrole{n}{begin\_time}\DUrole{o}{=}\DUrole{default_value}{None}}\sphinxparamcomma \sphinxparam{\DUrole{n}{end\_time}\DUrole{o}{=}\DUrole{default_value}{None}}}{}
\pysigstopsignatures
\sphinxAtStartPar
The base class of the trace processing. Implementations of objects of classes derived from the ProcessTrace
do some processing on traces defined in the derived classes initialization
\begin{quote}\begin{description}
\sphinxlineitem{Parameters}\begin{itemize}
\item {} 
\sphinxAtStartPar
\sphinxstyleliteralstrong{\sphinxupquote{trace}} (\sphinxstyleliteralemphasis{\sphinxupquote{ObsPy.Trace}}) \textendash{} The processed trace

\item {} 
\sphinxAtStartPar
\sphinxstyleliteralstrong{\sphinxupquote{begin\_time}} (\sphinxstyleliteralemphasis{\sphinxupquote{ObsPy.UTCDateTime}}) \textendash{} It limits the period, where a process is performed.
If begin\_time is not defined or it is earlier than the beginning of the trace,
the process is performed from the beginning of the trace

\item {} 
\sphinxAtStartPar
\sphinxstyleliteralstrong{\sphinxupquote{end\_time}} (\sphinxstyleliteralemphasis{\sphinxupquote{ObsPy.UTCDateTime}}) \textendash{} It limits the period, where a process is performed.
If end\_time is not defined or it is later than the end of the trace,
the process is performed to the end of the trace

\end{itemize}

\end{description}\end{quote}

\end{fulllineitems}

\index{get\_focal\_mechanism() (in module core.utils)@\spxentry{get\_focal\_mechanism()}\spxextra{in module core.utils}}

\begin{fulllineitems}
\phantomsection\label{\detokenize{api_core:core.utils.get_focal_mechanism}}
\pysigstartsignatures
\pysiglinewithargsret{\sphinxcode{\sphinxupquote{core.utils.}}\sphinxbfcode{\sphinxupquote{get\_focal\_mechanism}}}{\sphinxparam{\DUrole{n}{event}}\sphinxparamcomma \sphinxparam{\DUrole{n}{inversion\_type}\DUrole{o}{=}\DUrole{default_value}{None}}}{}
\pysigstopsignatures
\sphinxAtStartPar
Function get\_focal\_mechanism extracts the focal mechanism from the event.
If preferred\_focal\_mechanism\_id of the event is set it return the preferred focal mechanism.
Otherwise, it returns the first focal mechanism from the list.
The function is intended to extract the focal mechanism unconditionally and non\sphinxhyphen{}interactively.
Therefore, if preferred\_focal\_mechanism\_id is not set and there are multiple focal mechanisms,
the returned focal mechanism may be random.
\begin{quote}\begin{description}
\sphinxlineitem{Parameters}\begin{itemize}
\item {} 
\sphinxAtStartPar
\sphinxstyleliteralstrong{\sphinxupquote{event}} (\sphinxstyleliteralemphasis{\sphinxupquote{ObsPy.Event}}) \textendash{} The seismic event object

\item {} 
\sphinxAtStartPar
\sphinxstyleliteralstrong{\sphinxupquote{inversion\_type}} (\sphinxstyleliteralemphasis{\sphinxupquote{(}}\sphinxstyleliteralemphasis{\sphinxupquote{str}}\sphinxstyleliteralemphasis{\sphinxupquote{)}}) \textendash{} The name of tensor inversion type.
It must belong to the QuakeML MTInversionType category:
\sphinxcode{\sphinxupquote{\textquotesingle{}general\textquotesingle{}}}, \sphinxcode{\sphinxupquote{\textquotesingle{}zero trace\textquotesingle{}}}, \sphinxcode{\sphinxupquote{\textquotesingle{}double couple\textquotesingle{}}}, or None.

\end{itemize}

\sphinxlineitem{Returns}
\sphinxAtStartPar
The focal mechanism object or None if none focal\_mechanism is defined for the event
or the focal\_mechanism with the defined inversion type does not exist.

\sphinxlineitem{Return type}
\sphinxAtStartPar
ObsPy.FocalMechanism

\end{description}\end{quote}

\end{fulllineitems}

\index{get\_hypocentral\_distance() (in module core.utils)@\spxentry{get\_hypocentral\_distance()}\spxextra{in module core.utils}}

\begin{fulllineitems}
\phantomsection\label{\detokenize{api_core:core.utils.get_hypocentral_distance}}
\pysigstartsignatures
\pysiglinewithargsret{\sphinxcode{\sphinxupquote{core.utils.}}\sphinxbfcode{\sphinxupquote{get\_hypocentral\_distance}}}{\sphinxparam{\DUrole{n}{origin}}\sphinxparamcomma \sphinxparam{\DUrole{n}{station\_inventory}}}{}
\pysigstopsignatures
\sphinxAtStartPar
Function get\_hypocentral\_distance computes the local hypocentral distance in meters
from origin coordinates to station\_name coordinates.
The calculations do not take into account the curvature of the earth.
\begin{quote}\begin{description}
\sphinxlineitem{Parameters}\begin{itemize}
\item {} 
\sphinxAtStartPar
\sphinxstyleliteralstrong{\sphinxupquote{origin}} (\sphinxstyleliteralemphasis{\sphinxupquote{ObsPy.Origin}}) \textendash{} The ObsPy Origin object

\item {} 
\sphinxAtStartPar
\sphinxstyleliteralstrong{\sphinxupquote{station\_inventory}} (\sphinxstyleliteralemphasis{\sphinxupquote{ObsPy.Inventory}}) \textendash{} The station inventory object

\end{itemize}

\sphinxlineitem{Returns}
\sphinxAtStartPar
The hypocentral distance in meters and epicentral distance in degrees

\sphinxlineitem{Return type}
\sphinxAtStartPar
tuple(float, float)

\end{description}\end{quote}

\end{fulllineitems}

\index{get\_magnitude() (in module core.utils)@\spxentry{get\_magnitude()}\spxextra{in module core.utils}}

\begin{fulllineitems}
\phantomsection\label{\detokenize{api_core:core.utils.get_magnitude}}
\pysigstartsignatures
\pysiglinewithargsret{\sphinxcode{\sphinxupquote{core.utils.}}\sphinxbfcode{\sphinxupquote{get\_magnitude}}}{\sphinxparam{\DUrole{n}{event}}\sphinxparamcomma \sphinxparam{\DUrole{n}{magnitude\_type}\DUrole{o}{=}\DUrole{default_value}{None}}}{}
\pysigstopsignatures
\sphinxAtStartPar
Function get\_magnitude extracts the magnitude of the event.
If you want to extract a specific magnitude you can define it as magnitude\_type,
e.g. \sphinxcode{\sphinxupquote{get\_magnitude(event, magnitude\_type=\textquotesingle{}Mw\textquotesingle{})}}, otherwise, any magnitude will be extracted.
If the preferred\_magnitude\_id of the event is set it returns the preferred origin.
Otherwise, it returns the first magnitude from the list.
The function is intended to extract the magnitude unconditionally and non\sphinxhyphen{}interactively.
Therefore, if preferred\_magnitude\_id is not set and there are multiple magnitudes,
the returned origin may be random.

\sphinxAtStartPar
If event magnitude does not exist, but station\_name magnitudes exist, the new magnitude is computed
as the mean value of station\_name magnitudes.
\begin{quote}\begin{description}
\sphinxlineitem{Parameters}\begin{itemize}
\item {} 
\sphinxAtStartPar
\sphinxstyleliteralstrong{\sphinxupquote{event}} (\sphinxstyleliteralemphasis{\sphinxupquote{ObsPy.Event}}) \textendash{} The seismic event object

\item {} 
\sphinxAtStartPar
\sphinxstyleliteralstrong{\sphinxupquote{magnitude\_type}} (\sphinxstyleliteralemphasis{\sphinxupquote{str}}) \textendash{} (optional)
Describes the type of magnitude. This is a free\sphinxhyphen{}text. Proposed values are:
* unspecified magnitude (\sphinxcode{\sphinxupquote{\textquotesingle{}M\textquotesingle{}}}) \sphinxhyphen{} function search for exactly unspecified magnitude,
* local magnitude (\sphinxcode{\sphinxupquote{\textquotesingle{}ML\textquotesingle{}}}),
* moment magnitude (\sphinxcode{\sphinxupquote{\textquotesingle{}Mw\textquotesingle{}}}),
* energy (\sphinxcode{\sphinxupquote{\textquotesingle{}Energy\textquotesingle{}}}),
* etc.

\end{itemize}

\sphinxlineitem{Returns}
\sphinxAtStartPar
The magnitude object or None if the function cannot find or create the magnitude.
If only station\_name magnitudes exist, the new ObsPy Magnitude object is created,
but it is not appended to the event

\sphinxlineitem{Return type}
\sphinxAtStartPar
ObsPy.Magnitude

\end{description}\end{quote}

\end{fulllineitems}

\index{get\_net\_sta() (in module core.utils)@\spxentry{get\_net\_sta()}\spxextra{in module core.utils}}

\begin{fulllineitems}
\phantomsection\label{\detokenize{api_core:core.utils.get_net_sta}}
\pysigstartsignatures
\pysiglinewithargsret{\sphinxcode{\sphinxupquote{core.utils.}}\sphinxbfcode{\sphinxupquote{get\_net\_sta}}}{\sphinxparam{\DUrole{n}{name}}}{}
\pysigstopsignatures
\sphinxAtStartPar
Function get\_net\_sta extracts network and station\_name codes as strings
\begin{quote}\begin{description}
\sphinxlineitem{Parameters}
\sphinxAtStartPar
\sphinxstyleliteralstrong{\sphinxupquote{name}} (\sphinxstyleliteralemphasis{\sphinxupquote{str}}\sphinxstyleliteralemphasis{\sphinxupquote{ or }}\sphinxstyleliteralemphasis{\sphinxupquote{ObsPy.WaveformStreamID}}) \textendash{} The trace name. It can be the string or the WaveformStreamID object.
The text in the string is in the form ‘NN.SSS.LL.CCC’, where NN is the network code,
SSS is the station\_name code, LL is the location code, and CCC is the channel code.

\sphinxlineitem{Returns}
\sphinxAtStartPar
The tuple of the network code the station\_name code.

\sphinxlineitem{Return type}
\sphinxAtStartPar
tuple(str, str)

\end{description}\end{quote}

\end{fulllineitems}

\index{get\_origin() (in module core.utils)@\spxentry{get\_origin()}\spxextra{in module core.utils}}

\begin{fulllineitems}
\phantomsection\label{\detokenize{api_core:core.utils.get_origin}}
\pysigstartsignatures
\pysiglinewithargsret{\sphinxcode{\sphinxupquote{core.utils.}}\sphinxbfcode{\sphinxupquote{get\_origin}}}{\sphinxparam{\DUrole{n}{event}}}{}
\pysigstopsignatures
\sphinxAtStartPar
Function get\_origin extracts the origin from the event.
If preferred\_origin\_id of the event is set it return the preferred origin.
Otherwise, it returns the first origin from the list.
The function is intended to extract the event origin unconditionally and non\sphinxhyphen{}interactively.
Therefore, if preferred\_origin\_id is not set and there are multiple origins, the returned origin may be random
\begin{quote}\begin{description}
\sphinxlineitem{Parameters}
\sphinxAtStartPar
\sphinxstyleliteralstrong{\sphinxupquote{event}} (\sphinxstyleliteralemphasis{\sphinxupquote{ObsPy.Event}}) \textendash{} The seismic event object

\sphinxlineitem{Returns}
\sphinxAtStartPar
The origin (event location) object or None if none origin is defined for the event.

\sphinxlineitem{Return type}
\sphinxAtStartPar
ObsPy.Origin

\end{description}\end{quote}

\end{fulllineitems}

\index{get\_station\_id() (in module core.utils)@\spxentry{get\_station\_id()}\spxextra{in module core.utils}}

\begin{fulllineitems}
\phantomsection\label{\detokenize{api_core:core.utils.get_station_id}}
\pysigstartsignatures
\pysiglinewithargsret{\sphinxcode{\sphinxupquote{core.utils.}}\sphinxbfcode{\sphinxupquote{get\_station\_id}}}{\sphinxparam{\DUrole{n}{name}}}{}
\pysigstopsignatures
\sphinxAtStartPar
Function get\_station\_id extracts the station\_name name as a WaveformStreamID object
\begin{quote}\begin{description}
\sphinxlineitem{Parameters}
\sphinxAtStartPar
\sphinxstyleliteralstrong{\sphinxupquote{name}} (\sphinxstyleliteralemphasis{\sphinxupquote{str}}\sphinxstyleliteralemphasis{\sphinxupquote{ or }}\sphinxstyleliteralemphasis{\sphinxupquote{ObsPy.WaveformStreamID}}) \textendash{} The trace name. It can be the string or the ObsPy WaveformStreamID object.
The text in the string is in the form ‘NN.SSS.LL.CCC’, where NN is the network code,
SSS is the station\_name code, LL is the location code, and CCC is the channel code.

\sphinxlineitem{Returns}
\sphinxAtStartPar
The waveform stream object containing only the network code and the station\_name code.

\sphinxlineitem{Return type}
\sphinxAtStartPar
ObsPy.WaveformStreamID

\end{description}\end{quote}

\end{fulllineitems}

\index{get\_station\_name() (in module core.utils)@\spxentry{get\_station\_name()}\spxextra{in module core.utils}}

\begin{fulllineitems}
\phantomsection\label{\detokenize{api_core:core.utils.get_station_name}}
\pysigstartsignatures
\pysiglinewithargsret{\sphinxcode{\sphinxupquote{core.utils.}}\sphinxbfcode{\sphinxupquote{get\_station\_name}}}{\sphinxparam{\DUrole{n}{name}}}{}
\pysigstopsignatures
\sphinxAtStartPar
Function get\_station\_name extracts the station\_name name as a string
\begin{quote}\begin{description}
\sphinxlineitem{Parameters}
\sphinxAtStartPar
\sphinxstyleliteralstrong{\sphinxupquote{name}} (\sphinxstyleliteralemphasis{\sphinxupquote{str}}\sphinxstyleliteralemphasis{\sphinxupquote{ or }}\sphinxstyleliteralemphasis{\sphinxupquote{ObsPy.WaveformStreamID}}) \textendash{} The trace name. It can be the string or the WaveformStreamID object.
The text in the string is in the form ‘NN.SSS.LL.CCC’, where NN is the network code,
SSS is the station\_name code, LL is the location code, and CCC is the channel code.

\sphinxlineitem{Returns}
\sphinxAtStartPar
The string in the form ‘NN.STA’, where NN is the network code and SSS is the station\_name code.

\sphinxlineitem{Return type}
\sphinxAtStartPar
str

\end{description}\end{quote}

\end{fulllineitems}

\index{get\_units() (in module core.utils)@\spxentry{get\_units()}\spxextra{in module core.utils}}

\begin{fulllineitems}
\phantomsection\label{\detokenize{api_core:core.utils.get_units}}
\pysigstartsignatures
\pysiglinewithargsret{\sphinxcode{\sphinxupquote{core.utils.}}\sphinxbfcode{\sphinxupquote{get\_units}}}{\sphinxparam{\DUrole{n}{trace}}}{}
\pysigstopsignatures
\sphinxAtStartPar
Return the signal units of the trace
\begin{quote}\begin{description}
\sphinxlineitem{Parameters}
\sphinxAtStartPar
\sphinxstyleliteralstrong{\sphinxupquote{trace}} (\sphinxstyleliteralemphasis{\sphinxupquote{ObsPy.Trace}}) \textendash{} The trace object

\sphinxlineitem{Returns}
\sphinxAtStartPar
The string with units: ‘m/s’, ‘m/s\textasciicircum{}2’, or ‘m’, if the response was removed,
when in the processing\_parameters is the remove\_response process defined,
or ‘counts’ otherwise

\sphinxlineitem{Return type}
\sphinxAtStartPar
str

\end{description}\end{quote}

\end{fulllineitems}

\index{time\_ceil() (in module core.utils)@\spxentry{time\_ceil()}\spxextra{in module core.utils}}

\begin{fulllineitems}
\phantomsection\label{\detokenize{api_core:core.utils.time_ceil}}
\pysigstartsignatures
\pysiglinewithargsret{\sphinxcode{\sphinxupquote{core.utils.}}\sphinxbfcode{\sphinxupquote{time\_ceil}}}{\sphinxparam{\DUrole{n}{time}}\sphinxparamcomma \sphinxparam{\DUrole{n}{step}}}{}
\pysigstopsignatures
\sphinxAtStartPar
Returns the time rounded\sphinxhyphen{}up to the specified accuracy.
\begin{quote}\begin{description}
\sphinxlineitem{Parameters}\begin{itemize}
\item {} 
\sphinxAtStartPar
\sphinxstyleliteralstrong{\sphinxupquote{time}} (\sphinxstyleliteralemphasis{\sphinxupquote{ObsPy.UTCDateTime}}) \textendash{} The time object

\item {} 
\sphinxAtStartPar
\sphinxstyleliteralstrong{\sphinxupquote{step}} (\sphinxstyleliteralemphasis{\sphinxupquote{float}}) \textendash{} The accuracy units in seconds

\end{itemize}

\sphinxlineitem{Returns}
\sphinxAtStartPar
The new rounded\sphinxhyphen{}up time object

\sphinxlineitem{Return type}
\sphinxAtStartPar
ObsPy.UTCDateTime

\end{description}\end{quote}

\sphinxAtStartPar
Example:

\begin{sphinxVerbatim}[commandchars=\\\{\}]
\PYG{o}{\PYGZgt{}\PYGZgt{}} \PYG{k+kn}{from} \PYG{n+nn}{obspy}\PYG{n+nn}{.}\PYG{n+nn}{core}\PYG{n+nn}{.}\PYG{n+nn}{utcdatetime} \PYG{k+kn}{import} \PYG{n}{UTCDateTime}
\PYG{o}{\PYGZgt{}\PYGZgt{}} \PYG{k+kn}{from} \PYG{n+nn}{core}\PYG{n+nn}{.}\PYG{n+nn}{utils} \PYG{k+kn}{import} \PYG{n}{time\PYGZus{}ceil}
\PYG{o}{\PYGZgt{}\PYGZgt{}} \PYG{n}{time} \PYG{o}{=} \PYG{n}{UTCDateTime}\PYG{p}{(}\PYG{l+m+mi}{2024}\PYG{p}{,} \PYG{l+m+mi}{1}\PYG{p}{,} \PYG{l+m+mi}{3}\PYG{p}{,} \PYG{l+m+mi}{8}\PYG{p}{,} \PYG{l+m+mi}{28}\PYG{p}{,} \PYG{l+m+mi}{33}\PYG{p}{,} \PYG{l+m+mi}{245678}\PYG{p}{)}
\PYG{o}{\PYGZgt{}\PYGZgt{}} \PYG{n}{time\PYGZus{}ceil}\PYG{p}{(}\PYG{n}{time}\PYG{p}{,}\PYG{l+m+mf}{1.0}\PYG{p}{)}
\PYG{o}{\PYGZgt{}\PYGZgt{}} \PYG{n}{UTCDateTime}\PYG{p}{(}\PYG{l+m+mi}{2024}\PYG{p}{,} \PYG{l+m+mi}{1}\PYG{p}{,} \PYG{l+m+mi}{3}\PYG{p}{,} \PYG{l+m+mi}{8}\PYG{p}{,} \PYG{l+m+mi}{28}\PYG{p}{,} \PYG{l+m+mi}{34}\PYG{p}{)}
\PYG{o}{\PYGZgt{}\PYGZgt{}} \PYG{n}{time\PYGZus{}ceil}\PYG{p}{(}\PYG{n}{time}\PYG{p}{,}\PYG{l+m+mf}{60.0}\PYG{p}{)}
\PYG{o}{\PYGZgt{}\PYGZgt{}} \PYG{n}{UTCDateTime}\PYG{p}{(}\PYG{l+m+mi}{2024}\PYG{p}{,} \PYG{l+m+mi}{1}\PYG{p}{,} \PYG{l+m+mi}{3}\PYG{p}{,} \PYG{l+m+mi}{8}\PYG{p}{,} \PYG{l+m+mi}{29}\PYG{p}{)}
\PYG{o}{\PYGZgt{}\PYGZgt{}} \PYG{n}{time\PYGZus{}ceil}\PYG{p}{(}\PYG{n}{time}\PYG{p}{,}\PYG{l+m+mf}{0.1}\PYG{p}{)}
\PYG{n}{UTCDateTime}\PYG{p}{(}\PYG{l+m+mi}{2024}\PYG{p}{,} \PYG{l+m+mi}{1}\PYG{p}{,} \PYG{l+m+mi}{3}\PYG{p}{,} \PYG{l+m+mi}{8}\PYG{p}{,} \PYG{l+m+mi}{28}\PYG{p}{,} \PYG{l+m+mi}{33}\PYG{p}{,} \PYG{l+m+mi}{300000}\PYG{p}{)}
\PYG{o}{\PYGZgt{}\PYGZgt{}} \PYG{n}{time\PYGZus{}ceil}\PYG{p}{(}\PYG{n}{time}\PYG{p}{,}\PYG{l+m+mf}{0.01}\PYG{p}{)}
\PYG{o}{\PYGZgt{}\PYGZgt{}} \PYG{n}{UTCDateTime}\PYG{p}{(}\PYG{l+m+mi}{2024}\PYG{p}{,} \PYG{l+m+mi}{1}\PYG{p}{,} \PYG{l+m+mi}{3}\PYG{p}{,} \PYG{l+m+mi}{8}\PYG{p}{,} \PYG{l+m+mi}{28}\PYG{p}{,} \PYG{l+m+mi}{33}\PYG{p}{,} \PYG{l+m+mi}{250000}\PYG{p}{)}
\PYG{o}{\PYGZgt{}\PYGZgt{}} \PYG{n}{time\PYGZus{}ceil}\PYG{p}{(}\PYG{n}{time}\PYG{p}{,}\PYG{l+m+mf}{0.001}\PYG{p}{)}
\PYG{o}{\PYGZgt{}\PYGZgt{}} \PYG{n}{UTCDateTime}\PYG{p}{(}\PYG{l+m+mi}{2024}\PYG{p}{,} \PYG{l+m+mi}{1}\PYG{p}{,} \PYG{l+m+mi}{3}\PYG{p}{,} \PYG{l+m+mi}{8}\PYG{p}{,} \PYG{l+m+mi}{28}\PYG{p}{,} \PYG{l+m+mi}{33}\PYG{p}{,} \PYG{l+m+mi}{246000}\PYG{p}{)}
\end{sphinxVerbatim}

\end{fulllineitems}

\index{time\_ceil\_dist() (in module core.utils)@\spxentry{time\_ceil\_dist()}\spxextra{in module core.utils}}

\begin{fulllineitems}
\phantomsection\label{\detokenize{api_core:core.utils.time_ceil_dist}}
\pysigstartsignatures
\pysiglinewithargsret{\sphinxcode{\sphinxupquote{core.utils.}}\sphinxbfcode{\sphinxupquote{time\_ceil\_dist}}}{\sphinxparam{\DUrole{n}{time}}\sphinxparamcomma \sphinxparam{\DUrole{n}{step}}}{}
\pysigstopsignatures
\sphinxAtStartPar
Returns seconds from the time to the time rounded up to the specified accuracy.
\begin{quote}\begin{description}
\sphinxlineitem{Parameters}\begin{itemize}
\item {} 
\sphinxAtStartPar
\sphinxstyleliteralstrong{\sphinxupquote{time}} (\sphinxstyleliteralemphasis{\sphinxupquote{ObsPy.UTCDateTime}}) \textendash{} The time object

\item {} 
\sphinxAtStartPar
\sphinxstyleliteralstrong{\sphinxupquote{step}} (\sphinxstyleliteralemphasis{\sphinxupquote{float}}) \textendash{} The accuracy units in seconds

\end{itemize}

\sphinxlineitem{Returns}
\sphinxAtStartPar
The period in seconds to the rounded\sphinxhyphen{}up time

\sphinxlineitem{Return type}
\sphinxAtStartPar
float

\end{description}\end{quote}

\sphinxAtStartPar
Example:

\begin{sphinxVerbatim}[commandchars=\\\{\}]
\PYG{o}{\PYGZgt{}\PYGZgt{}} \PYG{k+kn}{from} \PYG{n+nn}{obspy}\PYG{n+nn}{.}\PYG{n+nn}{core}\PYG{n+nn}{.}\PYG{n+nn}{utcdatetime} \PYG{k+kn}{import} \PYG{n}{UTCDateTime}
\PYG{o}{\PYGZgt{}\PYGZgt{}} \PYG{n}{time} \PYG{o}{=} \PYG{n}{UTCDateTime}\PYG{p}{(}\PYG{l+m+mi}{2024}\PYG{p}{,} \PYG{l+m+mi}{1}\PYG{p}{,} \PYG{l+m+mi}{3}\PYG{p}{,} \PYG{l+m+mi}{8}\PYG{p}{,} \PYG{l+m+mi}{28}\PYG{p}{,} \PYG{l+m+mi}{33}\PYG{p}{,} \PYG{l+m+mi}{245678}\PYG{p}{)}
\PYG{o}{\PYGZgt{}\PYGZgt{}} \PYG{n}{time\PYGZus{}ceil\PYGZus{}dist}\PYG{p}{(}\PYG{n}{time}\PYG{p}{,}\PYG{l+m+mf}{0.1}\PYG{p}{)}
\PYG{l+m+mf}{0.054322}
\PYG{o}{\PYGZgt{}\PYGZgt{}} \PYG{n}{time\PYGZus{}ceil\PYGZus{}dist}\PYG{p}{(}\PYG{n}{time}\PYG{p}{,}\PYG{l+m+mf}{1.0}\PYG{p}{)}
\PYG{l+m+mf}{0.754322}
\end{sphinxVerbatim}

\end{fulllineitems}

\index{time\_floor() (in module core.utils)@\spxentry{time\_floor()}\spxextra{in module core.utils}}

\begin{fulllineitems}
\phantomsection\label{\detokenize{api_core:core.utils.time_floor}}
\pysigstartsignatures
\pysiglinewithargsret{\sphinxcode{\sphinxupquote{core.utils.}}\sphinxbfcode{\sphinxupquote{time\_floor}}}{\sphinxparam{\DUrole{n}{time}}\sphinxparamcomma \sphinxparam{\DUrole{n}{step}}}{}
\pysigstopsignatures
\sphinxAtStartPar
Returns the time rounded\sphinxhyphen{}down to the specified accuracy.
\begin{quote}\begin{description}
\sphinxlineitem{Parameters}\begin{itemize}
\item {} 
\sphinxAtStartPar
\sphinxstyleliteralstrong{\sphinxupquote{time}} (\sphinxstyleliteralemphasis{\sphinxupquote{ObsPy.UTCDateTime}}) \textendash{} The time object

\item {} 
\sphinxAtStartPar
\sphinxstyleliteralstrong{\sphinxupquote{step}} (\sphinxstyleliteralemphasis{\sphinxupquote{float}}) \textendash{} The accuracy units in seconds

\end{itemize}

\sphinxlineitem{Returns}
\sphinxAtStartPar
The new rounded\sphinxhyphen{}down time object

\sphinxlineitem{Return type}
\sphinxAtStartPar
ObsPy.UTCDateTime

\end{description}\end{quote}

\sphinxAtStartPar
Example:

\begin{sphinxVerbatim}[commandchars=\\\{\}]
\PYG{o}{\PYGZgt{}\PYGZgt{}} \PYG{k+kn}{from} \PYG{n+nn}{obspy}\PYG{n+nn}{.}\PYG{n+nn}{core}\PYG{n+nn}{.}\PYG{n+nn}{utcdatetime} \PYG{k+kn}{import} \PYG{n}{UTCDateTime}
\PYG{o}{\PYGZgt{}\PYGZgt{}} \PYG{k+kn}{from} \PYG{n+nn}{utils} \PYG{k+kn}{import} \PYG{n}{time\PYGZus{}floor}
\PYG{o}{\PYGZgt{}\PYGZgt{}} \PYG{n}{time} \PYG{o}{=} \PYG{n}{UTCDateTime}\PYG{p}{(}\PYG{l+m+mi}{2024}\PYG{p}{,} \PYG{l+m+mi}{1}\PYG{p}{,} \PYG{l+m+mi}{3}\PYG{p}{,} \PYG{l+m+mi}{8}\PYG{p}{,} \PYG{l+m+mi}{28}\PYG{p}{,} \PYG{l+m+mi}{33}\PYG{p}{,} \PYG{l+m+mi}{245678}\PYG{p}{)}
\PYG{o}{\PYGZgt{}\PYGZgt{}} \PYG{n}{time\PYGZus{}floor}\PYG{p}{(}\PYG{n}{time}\PYG{p}{,}\PYG{l+m+mf}{0.001}\PYG{p}{)}
\PYG{n}{UTCDateTime}\PYG{p}{(}\PYG{l+m+mi}{2024}\PYG{p}{,} \PYG{l+m+mi}{1}\PYG{p}{,} \PYG{l+m+mi}{3}\PYG{p}{,} \PYG{l+m+mi}{8}\PYG{p}{,} \PYG{l+m+mi}{28}\PYG{p}{,} \PYG{l+m+mi}{33}\PYG{p}{,} \PYG{l+m+mi}{245000}\PYG{p}{)}
\PYG{o}{\PYGZgt{}\PYGZgt{}} \PYG{n}{time\PYGZus{}floor}\PYG{p}{(}\PYG{n}{time}\PYG{p}{,}\PYG{l+m+mf}{0.01}\PYG{p}{)}
\PYG{n}{UTCDateTime}\PYG{p}{(}\PYG{l+m+mi}{2024}\PYG{p}{,} \PYG{l+m+mi}{1}\PYG{p}{,} \PYG{l+m+mi}{3}\PYG{p}{,} \PYG{l+m+mi}{8}\PYG{p}{,} \PYG{l+m+mi}{28}\PYG{p}{,} \PYG{l+m+mi}{33}\PYG{p}{,} \PYG{l+m+mi}{240000}\PYG{p}{)}
\PYG{o}{\PYGZgt{}\PYGZgt{}} \PYG{n}{time\PYGZus{}floor}\PYG{p}{(}\PYG{n}{time}\PYG{p}{,}\PYG{l+m+mf}{0.1}\PYG{p}{)}
\PYG{n}{UTCDateTime}\PYG{p}{(}\PYG{l+m+mi}{2024}\PYG{p}{,} \PYG{l+m+mi}{1}\PYG{p}{,} \PYG{l+m+mi}{3}\PYG{p}{,} \PYG{l+m+mi}{8}\PYG{p}{,} \PYG{l+m+mi}{28}\PYG{p}{,} \PYG{l+m+mi}{33}\PYG{p}{,} \PYG{l+m+mi}{200000}\PYG{p}{)}
\PYG{o}{\PYGZgt{}\PYGZgt{}} \PYG{n}{time\PYGZus{}floor}\PYG{p}{(}\PYG{n}{time}\PYG{p}{,}\PYG{l+m+mf}{1.0}\PYG{p}{)}
\PYG{n}{UTCDateTime}\PYG{p}{(}\PYG{l+m+mi}{2024}\PYG{p}{,} \PYG{l+m+mi}{1}\PYG{p}{,} \PYG{l+m+mi}{3}\PYG{p}{,} \PYG{l+m+mi}{8}\PYG{p}{,} \PYG{l+m+mi}{28}\PYG{p}{,} \PYG{l+m+mi}{33}\PYG{p}{)}
\PYG{o}{\PYGZgt{}\PYGZgt{}} \PYG{n}{time\PYGZus{}floor}\PYG{p}{(}\PYG{n}{time}\PYG{p}{,}\PYG{l+m+mf}{60.0}\PYG{p}{)}
\PYG{n}{UTCDateTime}\PYG{p}{(}\PYG{l+m+mi}{2024}\PYG{p}{,} \PYG{l+m+mi}{1}\PYG{p}{,} \PYG{l+m+mi}{3}\PYG{p}{,} \PYG{l+m+mi}{8}\PYG{p}{,} \PYG{l+m+mi}{28}\PYG{p}{)}
\end{sphinxVerbatim}

\end{fulllineitems}

\index{time\_floor\_dist() (in module core.utils)@\spxentry{time\_floor\_dist()}\spxextra{in module core.utils}}

\begin{fulllineitems}
\phantomsection\label{\detokenize{api_core:core.utils.time_floor_dist}}
\pysigstartsignatures
\pysiglinewithargsret{\sphinxcode{\sphinxupquote{core.utils.}}\sphinxbfcode{\sphinxupquote{time\_floor\_dist}}}{\sphinxparam{\DUrole{n}{time}}\sphinxparamcomma \sphinxparam{\DUrole{n}{step}}}{}
\pysigstopsignatures
\sphinxAtStartPar
Returns seconds from the time to the time rounded up to the specified accuracy.
\begin{quote}\begin{description}
\sphinxlineitem{Parameters}\begin{itemize}
\item {} 
\sphinxAtStartPar
\sphinxstyleliteralstrong{\sphinxupquote{time}} (\sphinxstyleliteralemphasis{\sphinxupquote{ObsPy.UTCDateTime}}) \textendash{} The time object

\item {} 
\sphinxAtStartPar
\sphinxstyleliteralstrong{\sphinxupquote{step}} (\sphinxstyleliteralemphasis{\sphinxupquote{float}}) \textendash{} The accuracy units in seconds

\end{itemize}

\sphinxlineitem{Returns}
\sphinxAtStartPar
The period in seconds to the rounded\sphinxhyphen{}down time

\sphinxlineitem{Return type}
\sphinxAtStartPar
float

\end{description}\end{quote}

\sphinxAtStartPar
Example:

\begin{sphinxVerbatim}[commandchars=\\\{\}]
\PYG{o}{\PYGZgt{}\PYGZgt{}} \PYG{k+kn}{from} \PYG{n+nn}{obspy}\PYG{n+nn}{.}\PYG{n+nn}{core}\PYG{n+nn}{.}\PYG{n+nn}{utcdatetime} \PYG{k+kn}{import} \PYG{n}{UTCDateTime}
\PYG{o}{\PYGZgt{}\PYGZgt{}} \PYG{n}{time} \PYG{o}{=} \PYG{n}{UTCDateTime}\PYG{p}{(}\PYG{l+m+mi}{2024}\PYG{p}{,} \PYG{l+m+mi}{1}\PYG{p}{,} \PYG{l+m+mi}{3}\PYG{p}{,} \PYG{l+m+mi}{8}\PYG{p}{,} \PYG{l+m+mi}{28}\PYG{p}{,} \PYG{l+m+mi}{33}\PYG{p}{,} \PYG{l+m+mi}{245678}\PYG{p}{)}
\PYG{o}{\PYGZgt{}\PYGZgt{}} \PYG{n}{time\PYGZus{}floor\PYGZus{}dist}\PYG{p}{(}\PYG{n}{time}\PYG{p}{,}\PYG{l+m+mf}{0.1}\PYG{p}{)}
\PYG{l+m+mf}{0.045678}
\PYG{o}{\PYGZgt{}\PYGZgt{}} \PYG{n}{time\PYGZus{}floor\PYGZus{}dist}\PYG{p}{(}\PYG{n}{time}\PYG{p}{,}\PYG{l+m+mf}{1.0}\PYG{p}{)}
\PYG{l+m+mf}{0.245678}
\end{sphinxVerbatim}

\end{fulllineitems}


\sphinxstepscope

\cleardoublepage
\begingroup
\renewcommand\chapter[1]{\endgroup}
\phantomsection


\chapter{References}
\label{\detokenize{bibliography:references}}\label{\detokenize{bibliography::doc}}
\begin{sphinxthebibliography}{1}
\bibitem[1]{bibliography:id13}
\sphinxAtStartPar
N. A. Haskell. Total energy and energy spectral density of elastic wave radiation from propagating faults. \sphinxstyleemphasis{Bulletin of the Seismological Society of America}, 54(6A):1811\textendash{}1841, 12 1964. URL: \sphinxurl{https://doi.org/10.1785/BSSA05406A1811}, \sphinxhref{https://arxiv.org/abs/https://pubs.geoscienceworld.org/ssa/bssa/article-pdf/54/6A/1811/5348868/bssa05406a1811.pdf}{arXiv:https://pubs.geoscienceworld.org/ssa/bssa/article\sphinxhyphen{}pdf/54/6A/1811/5348868/bssa05406a1811.pdf}, \sphinxhref{https://doi.org/10.1785/BSSA05406A1811}{doi:10.1785/BSSA05406A1811}.
\bibitem[2]{bibliography:id4}
\sphinxAtStartPar
James N. Brune. Tectonic stress and the spectra of seismic shear waves from earthquakes. \sphinxstyleemphasis{Journal of Geophysical Research (1896\sphinxhyphen{}1977)}, 75(26):4997\textendash{}5009, 1970. \sphinxhref{https://doi.org/10.1029/JB075i026p04997}{doi:10.1029/JB075i026p04997}.
\bibitem[3]{bibliography:id5}
\sphinxAtStartPar
James N. Brune. Correction {[}to "Tectonic Stress and the Spectra of Seismic Shear Waves from Earthquakes"{]}. \sphinxstyleemphasis{Journal of Geophysical Research}, 76(5):5002, 1971. \sphinxhref{https://doi.org/10.1029/JB076i020p05002}{doi:10.1029/JB076i020p05002}.
\bibitem[4]{bibliography:id14}
\sphinxAtStartPar
Leon Knopoff and Freeman Gilbert. Radiation from a strike\sphinxhyphen{}slip fault. \sphinxstyleemphasis{Bulletin of the Seismological Society of America}, 49(2):163\textendash{}178, 04 1959. URL: \sphinxurl{https://doi.org/10.1785/BSSA0490020163}, \sphinxhref{https://arxiv.org/abs/https://pubs.geoscienceworld.org/ssa/bssa/article-pdf/49/2/163/5303329/bssa0490020163.pdf}{arXiv:https://pubs.geoscienceworld.org/ssa/bssa/article\sphinxhyphen{}pdf/49/2/163/5303329/bssa0490020163.pdf}, \sphinxhref{https://doi.org/10.1785/BSSA0490020163}{doi:10.1785/BSSA0490020163}.
\bibitem[5]{bibliography:id10}
\sphinxAtStartPar
Paweł Wiejacz. Calculation of seismic moment tensor for mine tremors from the legnica\sphinxhyphen{}głogów copper basin. \sphinxstyleemphasis{Acta Geophysica Polonica}, XI(2):103\textendash{}122, 1992.
\bibitem[6]{bibliography:id9}
\sphinxAtStartPar
Grzegorz Lizurek, Jan Wiszniowski, Nguyen Van Giang, Beata Plesiewicz, and Dinh Quoc Van. Clustering and stress inversion in the song tranh 2 reservoir, vietnam. \sphinxstyleemphasis{Bulletin of the Seismological Society of America}, 107(6):2636\textendash{}2648, 09 2017. URL: \sphinxurl{https://doi.org/10.1785/0120170042}, \sphinxhref{https://arxiv.org/abs/https://pubs.geoscienceworld.org/ssa/bssa/article-pdf/107/6/2636/3992585/bssa-2017042.1.pdf}{arXiv:https://pubs.geoscienceworld.org/ssa/bssa/article\sphinxhyphen{}pdf/107/6/2636/3992585/bssa\sphinxhyphen{}2017042.1.pdf}, \sphinxhref{https://doi.org/10.1785/0120170042}{doi:10.1785/0120170042}.
\bibitem[7]{bibliography:id11}
\sphinxAtStartPar
Keiiti Aki and Paul G. Richards. \sphinxstyleemphasis{Quantitative Seismology}. University Science Books, 2 edition, 2002. ISBN 0935702962. URL: \sphinxurl{http://www.worldcat.org/isbn/0935702962}.
\bibitem[8]{bibliography:id12}
\sphinxAtStartPar
Sławomir Jerzy Gibowicz and Andrzej Kijko. \sphinxstyleemphasis{An introduction to mining seismology}. Academic Press, San Diego, 1 edition, 1994. ISBN 9780122821202. URL: \sphinxurl{https://api.semanticscholar.org/CorpusID:126723846}.
\end{sphinxthebibliography}


\renewcommand{\indexname}{Python Module Index}
\begin{sphinxtheindex}
\let\bigletter\sphinxstyleindexlettergroup
\bigletter{c}
\item\relax\sphinxstyleindexentry{core.signal\_utils}\sphinxstyleindexpageref{api_core:\detokenize{module-core.signal_utils}}
\item\relax\sphinxstyleindexentry{core.utils}\sphinxstyleindexpageref{api_core:\detokenize{module-core.utils}}
\indexspace
\bigletter{g}
\item\relax\sphinxstyleindexentry{green\_functions}\sphinxstyleindexpageref{api_lib:\detokenize{module-green_functions}}
\indexspace
\bigletter{s}
\item\relax\sphinxstyleindexentry{source\_models}\sphinxstyleindexpageref{api_lib:\detokenize{module-source_models}}
\item\relax\sphinxstyleindexentry{ssscat}\sphinxstyleindexpageref{api_run:\detokenize{module-ssscat}}
\item\relax\sphinxstyleindexentry{ssspy}\sphinxstyleindexpageref{api_run:\detokenize{module-ssspy}}
\indexspace
\bigletter{u}
\item\relax\sphinxstyleindexentry{utils}\sphinxstyleindexpageref{api_lib:\detokenize{module-utils}}
\end{sphinxtheindex}

\renewcommand{\indexname}{Index}
\printindex
\end{document}